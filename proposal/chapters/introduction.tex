\section{Introduction}
\label{sec:Introduction}
%\begin{itemize}
%	\item \emph{What is this research plan about?}
%    \item \emph{What are the (high-level) research gaps?}
%    \item \emph{What is the overall goal?}
%\end{itemize}


Visualizations are systematically related to the information that they represent 
(Bertin, 1983; Stenning \& Oberlander, 1995).
In scientific analysis, visualizations communicate complex information. We
attempt to determine the mathematical framework neccesary to design a dynamical
system to describe the dual-process of decision making. Dynamical systems provide us with
a way to recognize spacial patterns by using simulations. We intend to develop a
dynamic visualization of the decision process using a game-theoretical
framework. 

Analysis, topology, algebraic topology, and other fields of mathematics are building
in
the assumption that we can get to the end of infinity ($\infty$), extract
something, and take that something back into the mathematics \citep{NJ}.
In image processing, Grassman manifolds are used to proccess
sets of images \citep{MIC}. The space of a uncertainty states, restriced to nice
properties, is sometimes called an ensemble. This is the manifold that we wish to
address in our research, and it's construction is the main topic of this research proposal.
The cost function, which is crucial in game theory and mechanism design,
is not the focus of our analysis, in fact, it is not neccesary for fair
division of resources in mechanism design, as in \citep{MECH}. 
Through the use of retractions, the dual-process of transforming states using
exponential and logarithmic functions \citep{RMANI}, 
we are able to localize points of
interest on a manifold; these points transform to an algebra, where
properties are nicer, and therefore are computationally tractable. In
short, the projection of a manifold onto it's algebra allows for the
manifold to be locally smooth (at least $C^2$ continuous). We allow the
state to evolve on the manifold, and perform our calculations on the
algebra. Algebras are verstatile mathematical objects, and often
have a natural mapping to more complex fields, where we may build
regions of interest, basins of attraction, neighborhoods, i.e. balls, or
circles. From there we may construct bundles and bisectors \citep{NJ}. As
data science and its respective corrolations become more complex, so do
the geodesics on the corresponding manifold. Retractions generate approximations
of geodesics that are first-order accurate \citep{RMANI}.

We hope to design a type of manifold, one that functions as a
mechanism, similar to dicationary learning \citep{MIC}. We expect that the kernel
(core) is non-empty, and further, that new insight may be gained by the
visualization. We hope to make use of this manifold to further the security
game, and define a new color palette with a one-to-one correspondence to the
math model. We speculate that the manifold might be 'preshape' \citep{MIC}, as a
shape manifold is equal to a complex projective space or possibly
Grassman. We expect strong similarity, as well as fibration(s). The coefficients
will be iteratively calculated, and will provide scale, assisting in the
Visualization. The elements of the calculations are $n$-vectors in complex
space. What we hope is to generate a beautiful simulation.

