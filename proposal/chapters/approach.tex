\section{A Mathematical Approach to Evolve a Decision Process}
\label{sec:Approach}
%Short overview of subtopics.
%\subsection{Subtopic 1}
%\begin{itemize}
%	\item \emph{What approach will be used?}
%	\item \emph{Why is the approach promising?}
%	\item \emph{What are the expected results?}
%\end{itemize}

\subsection{Strategy space as a process}

Let $\mcS$ be a distribution of possible binary decisions in the strategy space
$\mcS \times \mcS$. The interaction between the strategies is modeled as an
arrival process, where we apply queuing theory to form the arrangement of
strategies. The cadlag (”continu a droite, limites a gauches”) functions are
right-continuous functions on with left limits. Defining $\mcS$ as a cadlag
finite variation process, we have that the quadratic variation equal to the sum
of the squares of the jumps, and so the mapping $\theta$ must preserve the
quadratic form. Thus, we preserve the ergodic properties of the jump-diffusion
process.


We assume a stochastic arrival process, and by associating the time of
arrival with a cost function and mapping $\theta$ to the complex plane, we
iteritavely create the geodesics necessary to build the SIG. This approach
allows for a topological analysis of the interacting strategies. A normalizing
function $\pi$ is assumed to keep the problem space within the range $\langle 0,
1 \rangle$, and choose our arrangement to allow one strategy $s$ to
stochastically dominate another. The exit process is determined by the noise
process, which ends the mapping $\theta$ and the strategy space $\mcS$ is
removed from the system.

We make use of mean field theorem (Xu and
Hajek ), and construct the SIG using an inverse mapping
from a sub-algebra revealed by the choice of arrangement on the arrival process.
We conjecture that if the subspace is simply connected, then there exists a
mapping $\phi$ from the kernel of the problem space.

The decision process is the result of pair interactions,
and can be represented by the index $(i,j)^n$ in the
resulting $2^n$ matrix. These pair interactions define the diffusion process,
and the evolution of the decision process for each strategy space.

\subsection{Queuing theory applied to an arrival process}

Let $\pi$ be a normalizing function acting on an angle $\theta \in \mathbb{C}$.
The mapping $\mathbb{N} \mapsto \mcK\times \mcS$ exists if and only if there is
the product space $\lbrace \pi k \mapsto r(\tau)\rbrace$, where $k=1$ defines
the mapping 
$$
    \theta(\cdot, k) \mapsto \mcP^k \times \mcK.
$$
$\mcK$ is the space of \emph{cadlag} functions restricted to
$\mathbb{A}^\mathbb{R} \times [0,1]^\mathbb{R}$,
and $\mcP$ is the projective space of the
player's strategy distribution $\mcS$. We define right-continuity as 
a \emph{stopping time} $\tau:\mcS \rightarrow [0,+\infty]^\mathbb{R}$, and
$r(\tau) \subset \mcS \times \mcS$ to be a subspace of size $\mcK$ at time $\tau$.

Let $\mcI$ be an arrangement on $\mcK$ where
$\theta (s_1, r(t)) > \theta (s_1, r(t+1))$ implies that $\pi k < \pi
(k+1)$ for all $t\in\tau$.
Then, fixing $t\in \tau$, let $\theta' \ne \theta$ be given by 
$$
    \theta' = \theta \cdot \displaystyle\sqrt{\frac{1+\beta^2}{1-\beta^2}},
$$
where $\beta = \tan\theta$. 

We claim that there exists a geodesic such that there is a
mapping from the origin, $0_{\mcS}$, is defined by the ball
$$
    B(\tau, \cdot) = \lbrace \lbrack s_i, s_{-i}\rbrack : i \in \mcI \subset
    \mcK\times \tau \rbrace,
$$
where $\lbrack \cdot, \cdot\rbrack : \mcS \times \mcS \mapsto \mcS$ is the Lie
bracket operator.

Suppose that $\langle s_i, s_{-i}\rangle < 0_{\mcS}$. We have a cone projection in
$\mcS_\tau$ that represents a bijection from $\pi\mcK \in \mcS \times \mcS$ to $r(\tau) \in \mcS$
such that
$B_{s_{-i}}(\tau) < \pi\mcK$ reveals a sub-algebra extending the strategy space
with respect to the real variable $\tau$, and is a homeomorphy with respect to
the kernel and the expectation $\mathbb{E}[r(\tau)]$. %2 tau^K 
Thus, the resulting subspace topology is simply connected.

We claim that if $s$ is in the null space of the ball, then there exists an identity
operator such that $\pi\kappa(s) \mapsto [s]$.
Now, given a function $\phi$, we examine the
extended strategy space where $s_1$ stochastically dominating $s_2$ implies that
$\mathbb{E}[\phi (\cdot, s_2 )] > \mathbb{E}[\phi (\cdot, s_1 )]$. Suppose
$\phi^{-1}$ preserves the quadratic form $(t^2- \langle s \rangle) \mapsto (t, r, \theta)$, so
that
$$
    \displaystyle \int_t^{t+1} \phi^{-1}(\theta(\mcK)) dt = \theta(\cdot,
    k+1)-\theta(\cdot,k).
$$

The collection of strategic decisions may be modeled by an arrival process
determined by the mapping $\phi(\tau, \cdot)$, giving the jump-diffusion
process
$$
    d\mcS_t = \theta(\cdot, t) \phi(\mcS_t) dt + \phi_t(\mcS_t) dW_t,
$$
where $W_t: [0, +\infty) \times \mcS \rightarrow \mcS$ is a one-dimensional
stop-time Brownian motion. 
As any cadlag finite variation process has quadratic variation equal to the sum
of the squares of the jumps $0=s_0<s_1<\cdots s_n$, the solution to
$$
    s_i(\tau) = \int_{\theta_i(s_i, t)}^{\theta_i(s_i,t+1)} \pi(\tau) \ d\tau
$$ 
for $t \in \tau$ gives a set of unique jumping points $0=s_0<s_1< \cdots < s_{\overline \mcK} = s_{MAX}$. 
Let each player's stop time $\tau$ occur at a random jumping point within their
strategy space, thereby fixing $\overline{\mcK}$ for that player.

\subsection{A topological bound on the expanded strategy space}

We consider a ball $B$, where a nonzero flux at the boundary represents an
uncertainty state, and take a random measure on $(\mcS \times \mcS, \mcK)$, i.e.
the $\sigma(\phi)$-algebra of the arrival process.
Then, each ball $B$ represents a distribution of possible strategies, and as subset of the
measure space we are able to compute its density function. 
Define the density operator $\rho$ on $\mcS\times \mcS$ as
$$
    \rho = \displaystyle\sum \vert s_i\rangle\langle s_{-i}\vert
$$
where $\vert s_i\rangle\langle s_{-i}\vert$ is the outer product. The state
$[s]$ is an algebra of $\mcS\times\mcS$, with expectation
value given by $\langle [s] \rangle = \text{tr}{\rho [s]}$, and is pure imaginary.
The density matrix used here is defined to be the statistical state of a system in
quantum mechanics, and is particularly useful in dealing with mixed states.
Define a graph $\mcG$ as the set subsets of $\mcS\times\mcS$ of fully connected nodes.
We claim that there exists an additional, induced metric, on $\mcG$.
The closed sphere of influence graph covers the intersections
of the closed balls $\lbrace\overline{B}\rbrace \subset \mcS\times \mcS$, where
$$
    \overline{B}= \lbrace B_i \in \mcS : \min\rho(s_i, s_{-i}) \le
    r_{\tau\times\tau} \rbrace.
$$
We finally claim to have an immersion in the surrounding
dynamical complex field $\mathbb{C}^{\tau\times\tau}$. The density matrix compresses
the space $\mcS \times \mcS$ to its canonical form. Thus, by Schur's lemma, the intertwining map
$phi^{-1} \mapsto \mcS \times \mcS$, is either $0$, or an isomorphism.
Thus,$\mcG$ is a unitary structure that can be seen as an orthogonal structure, a complex
structure, and a symplectic structure.

\subsection{The dynamics of the symplectic manifold}

The complex field $\mathbb{C}^{\tau\times\tau}$ encases the distributions of the
player strategies; that is, the gradient of the distribution across the boundary
of $\mcG$ determines the orientation of exterior. 
We proceed to determine the skew-distribution of the closed manifold. We build from the geometric form, and
extend the cross-product $[s_1 ]_x s_2$ , which is characterized by the
transition $s_i \tilde k$ to $s_i+1 \tilde k + 1$.

We examine the skew-Hermetian assignment, and the resulting tangent vector, or four-velocity. 
The four coordinate functions $\theta^c (\tau), c = 0, 1, 2, 3$ are real
functions of a real variable $\tau$. 
We have that $\phi\cdot \mcS\subset \mcS$ is the subset of skew-Hermetian matrices known as signed
permutation matrices. We extend our mean-field model to include this
representation by setting $\begin{bmatrix}0 \\ k\end{bmatrix} = \begin{bmatrix}1
\\ k\end{bmatrix}$ at the stop time $\tau$, where $\begin{bmatrix} \cdot \\
\cdot\end{bmatrix}$ is a binomial operator.

We claim that there exists an additional, induced metric, on the
SIG. The the closed sphere of influence graph (CSIG) covers the intersections
of the closed balls $\lbrace\overline{\mcS}\rbrace \subset \mcS\times \mcS$, where
$$
    \overline{S}_t= \lbrace S_t \in \mcS : \min\rho(S_t, S_{-t}) \le r_\tau \rbrace.
$$
Forming the CSIG,
we endow the resulting strategy space with the Minkowski metric $\eta$, defined by the matrix
$$
	(\eta)_\alpha\beta = \begin{bmatrix}
		-1 & 0 & 0 & 0 \\
		0 & 1 & 0 & 0 \\
		0 & 0 & 1 & 0 \\
		0 & 0 & 0 & 1 \\
	\end{bmatrix}.
$$
The Minkowski space $\mcM$ contains sub-manifolds endowed with a Riemannian metric yielding hyperbolic geometry. 
We conjecture that indefiniteness of the Minkowski metric will allow for the
construction of a utility function for the ensemble space. 
Consider a hyperbolic absolute risk aversion (HARA) $\mu$, which must adhere to,
for utility $\mu$,
$$
    \displaystyle\frac{\mu''(\langle s_i,s_{-i}\rangle)}{\mu'(\langle s_i,
    s_{-i})}.
$$

We conjecture that the marginal value of this construction maps back to the
convex kernel of the strategy space with a left limit. It is unclear if there is
a well-defined correspondence between the Minkowski metric and the HARA utility
function. The nature of the diffusion process is intended to allow for the
combination of strategies due to pair interactions occurring once the SIG is
completed, and will direct the evolution of the strategy space.

\subsubsection{CSIG Visualization}

The visualization of the CSIG may begin with simple arrangements.
As CSIGs always have a clique factor, i.e. for CSIG $G = \lbrace G_1,\cdots
G_n\rbrace$ there exists a spanning sub-graph whose connected components are always isomorphic to graphs in the set $\lbrace G_1,\cdots G_n\rbrace$. These connected components define the incident matrices of our dynamical network system. For example, suppose the node set $\mcX$ realizes $G$ in $\mcM$. Select a proper spanning forest $F$
of the $\mcM$-CSIG such that the sum of the lengths of the edges of $F$ is minimum, then, each connected component of $F$ must be a star.


