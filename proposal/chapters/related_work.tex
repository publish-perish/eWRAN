\section{Related Work}
\label{sec:Related Work}
%\begin{itemize}
%	\item \emph{Which major works consider a similar context?}
%	\item \emph{Which works are addressing same/similar problem and why are these works insufficient (Gaps in state of the art)?}
%	\item \emph{Which works use a similar methodology?}
%\end{itemize}
%Example references in parentheses format \citep{Raibert1986LeggedRobotsThatBalance, Vukobratovic2004ZeroMomentPoint} or as textual format as in \citet{Pratt1995SEA}.

\subsection{Background Studies}

As stated in \citet{VIS},
visualizations are systematically related to the information that they represent 
(Bertin, 1983; Stenning \& Oberlander, 1995).
Geodesics on a manifold are intensely complicated. Microsoft research has
well-publicized descriptions (Richard Hartley, 2016) detailing some of the
manifolds that we have found, i.e. $\mathbb{R}^n$, $\mcS^n$, the Lie group
($SO(3)$), positive definite matrices (or "covariance features"), Grassman
manifolds, Essential manifolds; each are used to capture a different aspect of
the geometry for analysis. For example, Shape manifolds capture the shape of an object. This
research has led algorithms design towards projective geometry, where retractions on
manifolds has allowed for a decrease in the computational complexity of solving
optimization problems \citet{MECH}. Different manifolds are useful
for solving different problems; the Lie group operator on $SO(3)$ may be used to
build a discrete extended Kalman filter to perform efficient computations
for rotation averaging. The visualizations of the output of these algorithms
are difficult to intuit, and require a high degree of specialization. We notice 
that none of these methods, as far as we know,
have been used to address a mixed strategy space (uncertainty space), and attempt to
visualize it. Algebras are versatile mathematical objects, and often
have a natural mapping to more complex fields, where we may build
regions of interest, basins of attraction, neighborhoods, i.e. balls, or
circles. From there we may construct bundles and bisectors (N.J.
Wilderberger, ????). It is natural to take these forms on an algebra and form a linear
system, such as a partitioning linear program. In the case that the
extreme point providing the solution to the system is integral, then the
associated \emph{game} is non-empty \citet{FAIR}. 
Thus, we arrive at a game-theoretical framework on which to begin an
analysis based on decision theory. Visualization cognition has been
studied as a subset of visual spacial reasoning, and steps have been
taken to build the association with dual-process systems, i.e. using both
mathematical modeling and heuristics \citet{VIS}. These
studies, however, focus on decision theory based on visualization
cognition, and draw conclusions from empirical studies. It is not
unreasonable to suppose that, in the very least, a
heuristic can be drawn from a mathematical model based on decision theory, and visualized.

It is known, particularly in computer vision, that kernels,
or similarity measures, are analogous to product spaces, which are the
dual space of quotient spaces. A symmetric kernel is equivalent to an inner
product. Filtering algorithms, such as the discrete 
extended Kalman filter on Lie
groups \citet{RMANI} function as a kernel
mechanism which is useful for averaging a sliding window of rotary
measurements. These types of mechanisms are often used in real-life
applications, where we often only have partial measurements. Partial
measurements are naturally uncertain, and correlate to partial knowledge in a
decision-making process. The mathematical model of decision theory with uncertainty 
was built for
simple Martingales, and the existence of Nash equilibria was shown for a
queuing network with a Poisson arrival process, which was
later extended to heterogeneous networks \citet{SUPS}.

The use of retractions on manifolds, and their efficiency at computing the
state of linear and non-linear systems results in massive computational
toolchains of every time. We address the need for a general metric and
associated model, that will support a game-theoretical analysis. A
sphere-of-influence (SIG) is both an incidence graph and a proximity graph,
where nice (local) properties on a manifold produce well-defined geometry, in
\citet{SIG}.
It remains to address the noise. Noisy partial measurements in the phase space
of the problem have been shown to converge for the Lorentz system in \citet{AVG}. 
As far as we know, the associated game has yet to be
modeled.

It remains to be mentioned that there are additional papers in the bibliography
which have yet to be integrated into the proposed topic, however, they have been
carefully selected for relevance and content.
