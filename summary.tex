\documentclass[10pt]{article}

\usepackage[text={6in,8.1in},centering]{geometry}
\usepackage{enumerate}
\usepackage{amsmath,amsthm,amssymb}
\usepackage{mathrsfs} % to use mathscr fonts

\newcommand{\mcL}{\mathcal{L}}
\newcommand{\mcI}{\mathcal{I}}
\newcommand{\mcM}{\mathcal{M}}
\newcommand{\mcP}{\mathcal{P}}
\newcommand{\mcX}{\mathcal{X}}
\newcommand{\mcC}{\mathcal{C}}
\newcommand{\mcS}{\mathcal{S}}
\newcommand{\mcD}{\mathcal{D}}
\DeclareMathOperator*{\argmax}{arg\,max}
\DeclareMathOperator*{\argmin}{arg\,min}

\newenvironment{block}{\begin{adjustwidth}{1.5cm}{1.5cm}\noindent}{\end{adjustwidth}}

\newtheorem{proposition}{Proposition}[section]
\newtheorem{theorem}{Theorem}[section]
\newtheorem{lemma}{Lemma}[section]
\newtheorem{corollary}{Corollary}[section]
\theoremstyle{definition}
\newtheorem{definition}{Definition}[section]


\begin{document}

\section{Introduction}
In this work, we focus on vulnerabilites to denial-of-service (DoS) threats in
a hostile network environment. We examine a worst-case scenario, maximizing the
exploitation of uncertainties inherent in the use of a spectrum sensing interval.
Our focus is on the behavior of the cognitive radio (CR) as used in customer
premises equipment (CPE), their behavior during the coexistence window, and
their use of the coexistence beacon protocol. According to the 802.22 draft, the
self-coexistence quiet periods are used for the specific purpose to detect
overlapping WRANs. The self-coexistence quiet periods are designed to support
dynamic resource sharing between the overlapping base stations (BSs), targeting
fair and efficient scheduling. We build our formulation on the specific case where the BS cells must
resort to adaptive on demand channel allocation. We provide an alternative
iteration of the process so that determines channels that can be acquired to satisfy the
QoS requirements of the given workloads. We determine that intra-channel sharing
is optimized by the cooperation, and the resulting inter-system demand alliviates the channel contention processes
with coexisting WRAN. In this work, we assume frame-by-frame contention, and
propose a naturally evolving organizational scheme that provides efficient and fair spectrum access of
available channels, and utilizes the cognitive interaction between secondary
users to detect, and precict, the presence of adversarial CR nodes.


The goal of our design is to ensure that the network may be safely used by 
formulating a CRN (WRAN) of cooperating base stations (BS) and supporting CPE.
Towards this goal, we develop a hybrid WRAN where cooperative games are played 
heirarchically, where depending on the player type and environment, the game 
is played in a centralized or decentralized structure.
We focus on three main issues in WRAN planning: (1) Spectrum allocation. A
network at equilibrium should maximize the throughput of the CRN through
intelligent use of base stations, (2) Quality of service (QoS). We consider QoS
to be a guarantee of a minimum rate of service while adhering to a priority
protocol. (3) Truthfulness in spectrum priority claims. 

We address a scenario where the standards assumed by the IEEE 802.22 draft for
wireless regional access networks (WRAN) are not supported. Namely, incomplete
provider (PU) data. This
scenario, as an example, could occur as a result of incomplete or corrupted
infrastructure. Further, we model the presence of a
powerful adversary and investigate WRAN operations in the worst case, where the
adversary has fully infiltrated the primary user's (PU) broadcast
spectrum. We propose a heirarchical game theoretic structure for resource allocation 
and investigate the ability of the WRAN to protect itself
and its cognitive nodes (CRs) from attack. We show that we are able to uphold
the desired property of self-coexistence, and provide a game-theoretic framework
and defense strategy to maintain network coherence with the desired QoS
converging to a dynamic, stable equilibrium. Finally, we present a Stackelberg
game and estimate the stability of our mechanism under duress.

CR nodes are advancing technologically (Huawei), and will have the
(communications) resources to collaborate on the allocation of spectra in
addition to determining open channels. 
As network use in sparse areas becomes more widespread, it is likely that a central dispatcher will not be able
to provide the desired QoS across the distributed CRN. The network at large will
remain sparse, which is a grand opportunity for malicious entities to target an
area and infiltrate the network without detection by a central policing agency.
Previous works have all focused on the protection of the PU. In our formulation,
we examine the interactions between the eCRs as a distributed network of
low-level, intelligent nodes.

\section{Related Work}

A branch-and-bound (B/\&B) algorithm for IEEE 802.22-based LTE networks has been
developed in a queue-based control (QBC), where resource allocation is
controlled by the queue size of nodes following their packet arrival
probabilities. They achieve optimal power and resource block assignment for each
mobile user, trading execution time and end-to-end packet delay. (CITE) 

Sakin and Razzaque solve the issue of self-coexistence as a large MIP problem,
which is known to be NP-hard, and several other approximations taking the
approach of nonlinear optimization (CITE).

With the 802.22 draft and the advent of orthogonal frequency modulation (OFM),
higher level algorithms began to form making use of the network capabilities.
Sengupta and Brahma (CITE) addressed the issues of efficiency and utilization
using a graph-theoretical approach. 


\section{The Model}

\subsection{The Evolutionary Game}

Before we arrive at the Stackelberg game, we make use of the flexibility of the
CRN to form a network topology based on the interconnectedness and resulting
strategies of the eCRs. In order to clarify our model, we use the CRN 
self-coexistence window as an update window to
a type of central limit order book (CLOB), which is owned and updated by the BS. 
The CLOBs contain the contended frames available during the self-coexistence window 
of each BS, and are shared among the BS via the eCRs within overlapping
areas. We form a cooperative game where the eCRs are responsible for forming queues in order
to optimize frame utilization of in the case where multiple BS must exist on the
same channel, defined in the 802.22 standard draft as on demand frame
contention. Our intention is to construct a
Baskett-Chandy-Muntz-Palacios (BCMP) network at equilibrium, where the eCRs form
queues to wait for frames from a single, or multiple BS. This BCMP network will simplify the spectrum allocation
with priority
problem for the BS, and alleviate contention in the self-coexistence window.
Each queue represents a different priority level, although players are allowed to choose any queue.
The supermarket game is relevant in scenarios where (1) the players choose which
queue to join without directions from a central dispatcher; (2) global workload or
queue length information is not available and customers randomly choose a finite
number of queues based on knowledge from their nearest neighbors; (3) there is a cost
associated with finding and then waiting in queues.

We have a set of CRN, each containing a set of FIFO queues. 
The queues formed at BSs that overlap, particularly within the overlapping areas 
of the individual CRN, may provide intelligence useful in enforcing
security. The 802.22 IEEE draft makes use of overlapping regions via the
contention beacon protocol. As in (CITE), we may assume that the BS randomly
selects CLUSTER head for each queue, who may aggragate or transmit data with
the BS and the other eCRs. Thus the importance of trust among queue members;
queues are natural targets for an adversary. 
The eCRs participate in a selection scheme that
encourages protected communities that practice proportional fairness. The queues
are self-organizing and
self-enforcing, as fitness is determined by the players,  players may apply peer
pressure to abide by regulations by affecting queue fitness.

The strategies of the eCRs, within a BCMP framework, allow us to complete the
metric on the space of cadlag functions restricted to $\mcL \in \mathbb{Z}^+$.
We represent the bijection $\mcL \rightarrow \mathbb{Z}^{n+}$ as the exterior of
the stationary set (probability distribution, but which one? density?) of the eCR strategies.
$n$-sphere, and construct a \emph{suspension} of the $(n-1)$-sphere representing
the boundary of a $(n+1)$-dimensional ball embedded in the equilibrium state of
the network. 
In order to construct a viable model for our 
final goal, we must have (1) self-configuration, and (2) automatic neighbor relations. 
We address our CRs as a finite set of
actions, and formalize the game play as sphere-of-influence (SIG)(CITE) graph.
To begin, we propose a set of mixed
strategies defined by a probability distribution over
the finite set of feasible strategies, and define a 
supermarket game played by the eCRs.

\subsubsection{The Supermarket Game}

Consider a single CRN containing $M$ heterogenous FIFO queues with unit exponential service rate and global
Poisson arrival rate $\lambda$. A CR completing a transmission at queue $i$ will either 
move to some new queue $j$ with (fixed) probability $P_{ij}$ or leave the system 
with probability $\displaystyle 1-\sum _{j=1}^{m}P_{ij}$, which is non-zero for some subset of the queues. 
In order to arrive at an advanced decision model, we first address three main ways to describe the choice $L_i$, which serves to characterize the
supermarket game as a dynamic interaction of intelligent players: (1) Choice structure,
(2) Preference maximization, and (3) Utility maximization.
Player $i$'s choice of $L_i$ queues forms a preferred set, the interactions between the eCRs in forming their
preferred set of queues, and the interactions between eCRs belonging to a specific
queue form a decision profile.We define each queue of of eCRs as a coalition within the CRN network including one or more
BS, where the decision is based on the expected probabilty distribution of
mixed strategies the eCRs.

Once a queue has been chosen, queue members jointly
compute a fitness metric which determines the internal valuation of the queue,
and is a factor in the queue's utility for all players. The fitness metric estimates how much
confidence the members of the queue have in each other. For each new member,
the residing eCRs in each queue asses the trustworthiness of the new member, after which the BS
updates its trust value including the new member. The more trustworthy the group
the better the overall fitness. A queue cannot refuse a new member, however, we
consider that any queue member can submit a participation request to the BS. In the case
that a new eCR causes the queue's fitness to drop below some threshold, a queue
member may choose to request that the BS participate in the fitness computation
before switching queues. 
Players joining the game choose a number of
queues to be sampled uniformly at random and joins the least loaded sampled
queue with respect to their valuation. Players are assumed to have a cost for both waiting and sampling, and they
want to minimize their own expected total cost. As in (CITE), we assume a unit exponential service rate of frames.

Denote a function $L_i(\cdot)$ from $[0, c_{MAX}]$ to
$\mcL$ as the strategy of player $i$. Following
standard game theoretic notation, all other players choose $L_{-i}$.
Further, let $\mcL = \lbrace 1,\cdots,L\rbrace$ be the set of
queues provided in the BS superframe. The expected total cost of player $i$ playing 
strategy $L_i$, with all other strategies fixed, is given by, 
$$
    C(L_i,L_{-i}) = c\mathbb{E}\lbrack W(L_i,L_{-i})\rbrack + \hat c L_i,
$$
where $\mathbb{E}\lbrack W(L_i,L_{-i})\rbrack$ is the
expected wait time, $c$ is the cost per unit waiting time, and $\hat c$ is the
(fixed) cost of sampling one queue. Note that if $s_1, s_2 \in \mcS$, with $s_1
<_{st} s_2$, then for all $L\in\mcL$, $\mathbb{E}[W(L,s_1)] >
\mathbb{E}[W(L,s_2)]$.
We have
that $\mcP(\mcL)$ is the set of all probability distributions over $\mcL$. The mixed
strategy $s_i(L_i) \in \mcS=\mcP(\mcL)$ is the probability that
eCR $i$ samples $L_i(\cdot)$ queues, 
the expected total cost for playing strategy $s_i$ is given by,
$$
    C(s_i, s_{-i}) = \displaystyle \sum_{L_i \in s_i} C(L_i,s_{-i})s(L_i).
$$

Adopting the mean field theorem from Xu and Hajek (CITE), we consider
the fitness of a queue using a rank-dependent utility model for
the expected fitness. Define the strategy set formed by $n$ eCRs waiting in queue $k$,
$Q_k = \lbrace L_1 \le L_2 \le\cdots \le L_n\rbrace \subset \mcL$.
Following the assuption that the eCRs will choose a strategy $L_i$
based on a potential distribution of losses or gains, the fitness function of
queue $k$ satisfies
$$
    W(Q_k) = \displaystyle\sum_{L_i\in Q_k} \pi_i u(L_i),
$$ 
where $\pi_i\in[0,1]$,
$u:\mcL \rightarrow\mathbb{R}$, and $\pi_i$ is a probability weight such
that $\sum_{j \in L_i}\pi_j = 1$. We assume monotonicity.
Restricting the sample space $\mcL\subset\mathbb{Z}^+$, suppose $\mcL_i(\cdot)\in \mcS$ is a given
non-decreasing step-function where the cost $L_i(c)$ is such that $c_1\le \cdots \le c_n$, and so $L_i(c) = l$ for $c_{l-1}\le c \le c_l$. 
Define a metric $d(L_1(\cdot), L_2(\cdot)) \equiv \vert\vert s_{L_1} -
s_{L_2}\vert\vert$. 

Denote the metric space of the mean field model as $\mcX$ such that $\mcL \subset \mcX$.
Now, let $\mcP(\mcL)$ denote the set of all probability distributions
over $\mcL$, and define a subset $S_i$ in $\mcP(\mcL)$ and a metric $\rho$.
There exists a bijection from $\mcL$ to $\mcP(\mcL)$.
The mean field equation describes a separable and complete metric space
$(\mcL, \mathbb{N}, \mcD)$, where $\mcD$ is the space of right-continuous,
left-limited functions (cadlag). Define,
$$
    r_i = \min\lbrace \rho(X_i, X_j) : j\ne i\rbrace, \qquad (i = 1, \cdots n)
$$
where $r_i$ denotes the minimum distance between $X_i$ and any other point in
$\mcX$. The open ball
$$
    B_i \lbrace X \in \mcX : \rho(X_i, X) < r_i\rbrace, \qquad (1 = 1, \cdots n)
$$
is the sphere of influence at $X_i$.

The mixed strategy vector $[s]$ forms a partition
function determined by the probabilities on $\mcS$. 
We define the priority of an eCR to be the energy function, where the
time-constrained players are more energetic than the cherry-picking types, as
they have a higher incentive to sample more queues.
The vector $[s] = [s_1, \cdots, s_N] \in \mcS\times\mcS$ is given by,
$$
    P_{s_i \in [s]} = \frac{1}{Z} e^{-\pi s},
$$
Restricting the sample space $\mcL\subset\mathbb{Z}^+$,
this is the length of the shortest queue.


For any pair interaction, 
$u_1 \le_{st} u_2$, indicate that $u_i$ is first-order stochastically
dominated by $u_j$, with respect to the norm. 
Now, fixing $t_0$, consider an initialization state, where no queues have been
sampled. We assume that queue fitness is i.i.d, as there
have been no pair interactions. 
Let $i$ be a queue containing $n$ eCRs. The $n$-dimensional sphere $S^n$
of radius $r$ in $(n+1)$-dimensional Euclidean space $\bf{E}^{n+1}$, the
Riemann metric induced by $\bf{E}^{n+1}$ is given by,
$$
    g = \frac{4r^4(d\phi^2 + d\psi^2)}{(r^2 + \phi^2 + \psi^2)^2}.
$$

Fix $s_i(\cdot) \in \mcL$, and suppose all other players use mixed strategy
$s_{-i}(\cdot)$, the expected fitness of queue $k$, 
$$
   C(s_i, s_{-i}) = c\displaystyle\sum_{[s]_i} G(L_i, s_{-i}) + \hat c L_i,
$$
where $G(L,s)\mapsto \mathbb{R}$ is a stereographic projection of the
exterior and interior of the manifold $\mcS(\mcL)$. We define the cost of an eCR
joining a queue as a direct sum *****

{\lemma{The Enveloping Algebra}\\
Let $\alpha_1, \alpha_2$ be random measures on the $\sigma$-algebra of the
Poisson arrival process.
There exists an arrangement $u_1<u_2<\cdots u_N$ of priorities in a queue if
and only if there exists a function $\phi$ such that $\phi(\alpha_1) =
u((-\infty,\alpha_k]), \ k = 1,2,\cdots, N$ and $\phi(\alpha_1) <
\phi(\alpha_2) < \cdots < \phi(\alpha_N)$.
}\\
\textbf{Proof:} OUTLINE : (1) $\alpha_k$ is the CDF of 
Thus, we define $\mcM$ and and $\mcP$ to be the measures and probability measures,
respectively, on the solution space.


TODO: 
1. show gradient in each sphere from type distribution,
2. parameterize the hyperboloid
- use tangent space
3. complete the exterior, must be convex
4. define the incidents on the SIG

We consider the division of contended frames during the contention beacon
protocol. In this priority FIFO scheme, each eCR will
choose the queue with the best priority match. Following the lead of Lal
and Rao (1997), we define two types of eCR nodes: cherry pickers and
time-constrained. Cherry pickers are known to have a lower opportunity cost, and
prefer to maximize their utility in terms of sustainable bandwidth. 
Time-constrained players make up the majority, attributing a higher weight to
the cost of selecting a queue. Letting one player type act as a sink, while the other
is a source, we construct a gradient,
$$
    e
$$
{\lemma{The Hyperbolic Interior}\\ 
Let $s_1 <_{st} s_2 \in \mcS\times \mcS$ be mixed strategies.
There exists a transform $\gamma: [0,1] \rightarrow [0,1]$ such that 
$$
\gamma\bigg( \displaystyle\sum_{n \in \vert \mcL\vert} \pi_n\bigg ) =
\gamma(\omega) + \gamma(1-\omega) = 1,
$$
where $\omega$ is the percentage of eCRs of the cherry-picking type.
}\\
\textbf{Proof:} NEED OUTLINE
We construct $g(L,s)$ as follows: 
1. projection onto big sphere (exp)
2. limits being the map projection onto (sphere, hyperboloid)

We **** determined by a holomorphic function.\\
Then, defining  ********** the equilibrium state probabilities are given by
$$
    \displaystyle \pi (x_{1},x_{2},\ldots ,x_{m}) = \kappa \pi_{1}(x_{1}) \pi _{2}(x_{2})\cdots \pi_{m}(x_{m}),
$$
where $\kappa$ is a normalizing constant, and $\pi_i(\cdot)$ represent the
equilibrium distribution for queue $i$. Thus, the CRN may be represented by a
BCNP network, where
$$
    {\displaystyle \scriptstyle {\pi _{i}(\cdot )}} {\displaystyle \scriptstyle {\pi _{i}(\cdot )}}
$$
represents the equilibrium distribution for queue $i$.



\subsection{Enhanced Cognitive Structure}
The ability of the CRN to reconfigure based on cognition of the environment
motivates us to construct our model as a study of momentum in a network between eCRs. 
We can think of this quantity as a measurement of flux passing through the
boundary of the ball $B_i$, although it is a positive quantity, the exterior of
the ball is a field with the same measure, and so although the field may have
velocity, there is no motivating factor as equilibrium is achieved at stasis.
Now define, for a player $i$, a function
$\alpha_t(\mcS) \longrightarrow S_i$ as the queues in $\mcS$ in the
complex field that are connected to player $i$ at time $t$. 
Thus we model the pair interaction, and define the the player's initial utility
function as a hyberbolic absolute risk aversion (HARA), and so must adhere to,
for utility $u$,
$$
    \displaystyle\frac{u''(C(L_i,L_{-i}))}{u'(C(L_i,L_{-i}))}.
$$
The initial utility for two eCRs is dependent on their type; cherry-pickers
(eCR$_{\omega}$), who
avoid the risk of oversampling, and time constrained eCRs, will try to
find the shortest queue, with higher costs (eCR$_{1-\omega}$).
For any player $i$, we take the cost of the wait time to be a nondegenerate
continuous probability density function as $\int_0^{c_{MAX}} f(c) \  dc = 1$, 
where $f$ is any cadlag function.
We begin by examining the interactions between eCRs, the game begins without any
player holding any bias towards any other player, as no queues have been
sampled and no pair interactions have occured. The utility of the players 
is function of service time, we assume an exponential utility:
$$
    u(\alpha) = 1 - e^{-\gamma \alpha},
$$
where $\alpha$ is the \emph{externality} associated with the on the trustworthiness of the
participating players, or in other words, $\alpha$ is an indication of the
efficiency of the queue when considering only the overhead incurred by pair
communications within a queue. Define the total
variational norm, where $f$ is any cadlag function,
$$
    \vert\vert u\vert\vert_{TV} = \sup\lbrace \int f \ du : \vert\vert
    f\vert\vert_\infty \le 1 \rbrace.
$$

\subsubsection{Sphere of Influence Graph (SIG)}
 We define the sphere of influence graph
(SIG) as a vertex set $\mcX$ and edge set 
$$
\lbrace\lbrack X_i, X_j\rbrack : B_i \cap B_j \ne 0, i\ne j\rbrace.
$$
As SIG is i\emph{simultaneously} a proximity graph and an intersection graph, we
may look for its closure in order to complete our construction. The closed SIG
graph forms a complete topology with respect to the preference of its central
eCR, within which we may make behavioral predictions. We define the SIG for an
eCR $i$, $SIG_i$ to be a set of vertices surrounding $i$, where another vertex
$j\in SIG_i$ probability of intersection is nonzero if and only if its core
$C_j$, is a subset of $SIG_i$ ((CHECK!). An intersection is defined to
be an interaction between the two eCRs (vertices) within a queue (CHECK!).

We may view the strategy space of eCR $i$ as a pair $(\mcL, \mathbb{P}(\mcS))$
associating probabilty with a graded manifold. We define a motivating operator $\phi$ to be a
function of the probability current. Define $\varepsilon^*$ to be the complex
conjugate of the potential energy of the player in stasis, $\varepsilon$.
That is, the
collection of balls $\mathbb{B}_{i=1\cdots n}$ who share an edge set at time $t$.
$$
\vert \mu_i \vert \ge \mu_k \ \forall \ s_i \in S_i, s_k \in \mcP(\mcL).
$$
Each ball represents a distribution of possible strategies, and as subset of the
measure space, we are able to compute its density function. Now, we provide a
formal characterization of the eCR as defined by its player type. Consider a
player $i$ with associated ball $B_i$ and graph $SIG_i$. Also consider that the ball $B_i$ has nonzero flux
at the boundery, and so $i$ is at the same time in an uncertain state. The
density matrix used here is defined to be the statistical state of a system in
quantum mechanics, and is particularly useful in dealing with mixed states. Define the measure $\rho$ on our player's strategy space as
$$
\rho = \displaystyle\sum p_i\vert\kappa_i\rangle\langle\kappa_i\vert
$$
where $\vert\kappa_i\rangle\langle\kappa_i\vert$ is the outer product. Now,
equipped with a proper embedding in the surrounding
dynamical complex field, player $i$'s type may be represented by a statistical
ensemble (mixed type), with probability $p_i$ that the system is in the pure
state $\vert \kappa$. We begin a tesselation process on $SIG_i$, a collection of
mobius triangles which are oriented perpendicular to the direction of flux, and
define a map projection $\varphi(SIG(B_I))$ as,

We claim that there exists an additional, induced metric, on the
SIG. The the closed sphere of influence graph $SIG_i$ is defined by the intersections of the closed balls
$$
\overline{B}_i= \lbrace X \in \mcX : \rho(X_i, X) \le r_i\rbrace.
$$
We define the closed graph $\overline{SIG_i}$ 

Let $\Lambda$ define the class of strictly increasing continuous mappings of
$[0,1]$ that are self-onto, i.e. self-coherent. For any $\lambda \in \Lambda$,
define the unit exponential function to be a mapping from the space of
continuous probabilty under uncertainty to the discrete possible space. Thus, we
define the metric, 
$$
\vert \lambda \vert = \sup \limits_{u\rightarrow s}\frac{\lambda\rightarrow u}{s}.
$$
The resulting topology is invariant under discrete possibility, and allows for
the weak convergence of our cadlag functor. Consider the infintesimal interval, 
$$
dS^2 = c^2 - dV^2,
$$
of strategies.

TODO:
\begin{enumerate}
\item Create hyperbola in Minkowski space
\item Linearize node types using retraction (almost homogenous) using Lorentz
time-invariance 
\item Partition function to the product space
\item Start with identical particles (blank slate)
\item Define progression of belief (use the alleles?)
\item Mutate to generate attacks with fixed binomial dist (of attack type)
\end{enumerate}


CLAIM: We now show that the process cycle of the eCR adhers to the evolutionary
model.
\begin{itemize}
\item The manifold of the eCRs network is factorially bounded above by its
most exterior exponent, i.e. there exists a calibration on $\mcX$,
$$
\frac{\omega^k}{k!},
$$    
and therfore, the topology of the sublayer of the CRN is a simplectic
manifold $(\mcX,\omega)$. 
\item The eCRs are type invariant.
\item The exterior is continuous on the field.
\end{itemize}


The mobius triangles are subject to a piecewise skew transform, and
define a mapping function. We may assume that the the decision framework of
the eCRs is a skew function with respect to the flux conditions on the boundary
of
the surrounding field. 
{
\lemma{Existence and Contraction of a Kahler Manifold/ Mean Field Chaoticity}
}
\textbf{Proof:}
The fully connected and closed queueing system is fully
embedded in the frequency space. The emedding, along with the uncertainty
overlay allows us to formally represent the heterogenous eCRs. 
We will show that the decoupling of a distributed competitive game 
and the resulting niche market, a deformation of the space as the distribution
of density in the SIG manifold tends towards the central limit. Consider the
connected map generated on the skew plane, the flow of the phase space will
is diverted by the skew force, and may not lie on the outer hull, breaking
its convexity and introducing a nonlinear component to the smooth hull as
the tensor flow interacts with the surrounding field.
(FINISH)
\emph{Corollary:}
Every complex submanifold of a Kahler mainfold is volume minimizing in its
homology class.

TODO:
1. model the flux as the magnetic
component of the phase induced by the boundary potential (will be a plane)


Let $\mu_i \in \mcP(\mcL)$ to be the mixed strategy
of player $i$, in other words, $\mu(L_i)$ is the probablility that player $i$
chooses $\langle L_k \rangle_{k\in S_i}$ queues. Let player $i$ be a time-constrained player, and fix all player strategies.

Then, the mean field equations for the equilibrium distribution is given by,
$$
C(L_i, L_{-i} = c\mathbb{E}\big\lbrack W(L_i(c),L_{-i})\big\rbrack + c_s L_i(c).
$$
The player wishes to minimize thier expected total cost by finding the optimal $L_i(\cdot)$. 

\section{Survivability}

In this section, we focus on vulnerabilites to denial-of-service (DoS) threats in the hostile network environment due to the uncertainties in the licensed user detection made during the spectrum sensing interval.
Our model assumes the existence of a hostile network environment, defined by the
presence our threat models, consisting of two types of attack: explicit or
implicit attacks. Explicit attacks include jamming, masking... Defense against
an explicit attack requires that the network detect an adversary hiding in the
PU spectrum. Implicit attacks are concerned with corruption of the decision
making process of the eCR. 
operating as an agent of the PU. As the adversary nodes are agents of the PU,
they are assumed to have knowledge of the characteristics of the PU's signal in
advance. The adverserial CPEs (ACPEs) therefore have an advantage as they do not
need to approximate the PU signal for detection. The adversary nodes are
classified according to their attacks and modeled as a Bayesian attack graph
(BAG). Describing the CRN similarly, we allocate defensive attributes to the eCR network graph, effectively forming an integrated defense system (IDS). We then form a Stackelberg game and derive bounds on the stability/ survivablility of the CRN.
W evaluate the performances of the IDS using the Positive Predictive Value (PPV), which is the probability that an alert raised by the IDS represents an actual intrusion. 
We begin with an introduction to the Stackelberg game from a topological
perspectiv game from a topological perspective, and show that a SOLUTION exists
and is a solution to the heirarchical game.

\subsection{The Stackelberg Topology}

(EDIT)

\end{document}
