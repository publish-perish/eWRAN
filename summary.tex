\documentclass[10pt]{article}

\usepackage[text={6in,8.1in},centering]{geometry}
\usepackage{enumerate}
\usepackage{amsmath,amsthm,amssymb}
\usepackage{mathrsfs} % to use mathscr fonts

\newcommand{\mcL}{\mathcal{L}}
\newcommand{\mcA}{\mathcal{A}}
\newcommand{\mcI}{\mathcal{I}}
\newcommand{\mcM}{\mathcal{M}}
\newcommand{\mcP}{\mathcal{P}}
\newcommand{\mcX}{\mathcal{X}}
\newcommand{\mcC}{\mathcal{C}}
\newcommand{\mcG}{\mathcal{G}}
\newcommand{\mcS}{\mathcal{S}}
\newcommand{\mcD}{\mathcal{D}}
\newcommand{\mcB}{\mathcal{B}}
\newcommand{\mcW}{\mathcal{W}}
\DeclareMathOperator*{\argmax}{arg\,max}
\DeclareMathOperator*{\argmin}{arg\,min}

\newenvironment{block}{\begin{adjustwidth}{1.5cm}{1.5cm}\noindent}{\end{adjustwidth}}

\newtheorem{proposition}{Proposition}[section]
\newtheorem{theorem}{Theorem}[section]
\newtheorem{lemma}{Lemma}[section]
\newtheorem{corollary}{Corollary}[section]
\theoremstyle{definition}
\newtheorem{definition}{Definition}[section]


\begin{document}

\section{Introduction}
In this work, we propose a mechanism to direct the distribution of contended
cells that will reveal the priority of a secondary cognitive radio user (CR).
Additionally, we model a learning process unique to the fluid and distributed
nature of the IEEE 802.22 network (WRAN). We model a cognitive radio network
(CRN) as a game among CRs, and maximize the utility of a set of overlapping base
stations (BS).
According to the 802.22 draft, the
self-coexistence quiet periods are used for the specific purpose to detect
overlapping WRANs, and are designed to support
dynamic resource sharing between the overlapping base stations (BSs), targeting
fair and efficient scheduling. We build our formulation on the specific case
where the BS cells must resort to adaptive on demand channel allocation. 
We examine a worst-case scenario, maximizing the
exploitation of uncertainties inherent to a spectrum sensing period.
Our focus is on the behavior of the cognitive radio during the
coexistence window, and their use of the coexistence beacon protocol. 
We provide an alternative
iteration of the process so that determines channels that can be acquired to satisfy the
QoS requirements of the given workloads. We determine that intra-channel sharing
is optimized by the cooperation, and the resulting inter-system demand
alliviates the channel contention processes with coexisting CRN. 
As network use in sparse areas becomes more widespread, it is likely that a central dispatcher will not be able
to provide the desired QoS across the distributed CRN. 

We consider the division of contended frames during the contention beacon
protocol. We propose a priority network policy, determined by a queuing network for frame
allocation. The CRs will choose the queue with the best priority match. 
We focus on three main issues in WRAN planning: (1) Spectrum allocation. A
network at equilibrium should maximize the throughput of the CRN through
intelligent use of base stations, (2) Quality of service (QoS). We consider QoS
to be a guarantee of a minimum rate of service while adhering to a priority
protocol. (3) Truthfulness in spectrum priority claims. 
We show that we are able to uphold
the desired property of self-coexistence, and model
a defense strategy as a Stackelberg game between the CRs and an adversary type
CR node with arrival rate and strategy as a directed mutations
of player type coefficients. Our goal is to maintain network coherence 
with the desired QoS converging to a dynamic, stable equilibrium. 

\section{Related Work}

A branch-and-bound (B/\&B) algorithm for IEEE 802.22-based LTE networks has been
developed in a queue-based control (QBC), where resource allocation is
controlled by the queue size of nodes following their packet arrival
probabilities. They achieve optimal power and resource block assignment for each
mobile user, trading execution time and end-to-end packet delay. (CITE) 

CR nodes are advancing technologically (Huawei), and will have the
(communications) resources to collaborate on the allocation of spectra in
addition to determining open channels, as in cooperative spectrum sensing
(CITE). 

Sakin and Razzaque solve the issue of self-coexistence as a large MIP problem,
which is known to be NP-hard, and several other approximations taking the
approach of nonlinear optimization (CITE).

With the 802.22 draft and the advent of orthogonal frequency modulation (OFM),
higher level algorithms began to form making use of the network capabilities.
Sengupta and Brahma (CITE) addressed the issues of efficiency and utilization
using a graph-theoretical approach. 

The mean field model was shown for general i.i.d random process, 
In [CITE] it was proven that there exists a Nash equilibrium in the
steady-state using Martingales.

\section{The Model}

\subsection{The Evolutionary Game}

In order to clarify our model, we use the CRN 
self-coexistence window header frame as an update window to
a type of central limit order book, which is owned and updated by the BS. 
The priority assignments from the eCRs waiting for contended frames 
from the BS are shared within the CRN via the eCRs within overlapping
areas. We form a cooperative game where the eCRs are responsible for forming queues in order
to optimize frame utilization of in the case where multiple BS must exist on the
same channel, defined in the 802.22 standard draft as on demand frame
contention. Our intention is to construct a queuing network at equilibrium, where the eCRs form
queues to wait for frames from a single, or multiple BS. 
We use a supermarket game, which is relevant in scenarios where (1) the players choose which
queue to join without directions from a central dispatcher; (2) global workload or
queue length information is not available and customers randomly choose a finite
number of queues based on knowledge from their nearest neighbors; (3) there is a cost
associated with finding and then waiting in queues.

We have a set of CRN, each containing a set of queues, one assigned to each BS. 
The queues formed at BSs that overlap cooperate intelligently. 
The 802.22 IEEE draft makes use of overlapping regions via the
contention beacon protocol. As in (CITE), we may assume that the BS randomly
selects a cluster head for each queue, who may aggragate or transmit data with
the BS and the other eCRs. Thus the importance of trust among queue members;
queues are natural targets for an adversary. 
The eCRs participate in a selection scheme that
encourages cooperation by a shared fitness function. The queues
are self-organizing and
self-enforcing; players may apply peer
pressure to abide by regulations by affecting queue fitness.

Following the lead of Lal
and Rao (1997), we define two types of eCR nodes: cherry pickers and
time-constrained. Cherry pickers are known to have a lower opportunity cost, and
prefer to maximize their utility in terms of sustainable bandwidth. 
Time-constrained players make up the majority, attributing a higher weight to
the cost of selecting a queue. 
In order to construct a viable model for our 
final goal, we must have (1) self-configuration, and (2) automatic neighbor relations. 
We address our CRs as a finite set of
actions, and formalize the game play as sphere-of-influence (SIG)(CITE) graph.
To begin, we propose a set of mixed
strategies defined by a probability distribution over
the finite set of feasible strategies, and define a 
supermarket game played by the eCRs.

\subsubsection{The Supermarket Game}

Consider a single CRN containing $M$ heterogenous FIFO queues with unit exponential service rate and global
Poisson arrival rate $\lambda$. A CR completing a transmission at queue $i$ will either 
move to some new queue $j$ with (fixed) probability $P_{ij}$ or leave the system 
with probability $\displaystyle 1-\sum _{j=1}^{m}P_{ij}$, which is non-zero for some subset of the queues. 
In order to arrive at an advanced decision model, we first address three main ways to describe the choice $L_i$, which serves to characterize the
supermarket game as a dynamic interaction of intelligent players: (1) Choice structure,
(2) Preference maximization, and (3) Utility maximization.
Player $i$'s choice of $L_i$ queues forms a preferred set, the interactions between the eCRs in forming their
preferred set of queues, and the interactions between eCRs belonging to a specific
queue form a decision profile. We define each queue of of eCRs as a coalition
within the CRN network including one or more BS, where the decision is based on
the expected probabilty distribution of the mixed strategies the eCRs.

The rules of the queue are as follows:
A queue cannot refuse a new member, however, we
consider that any queue member can submit a participation request to the BS.
Players joining the game choose a number of queues to be sampled uniformly at random and joins the best queue with respect
to the strategies of the other players. 

Denote a function $L_i(\cdot)$, from
$[0, c_{MAX}]$ to $\mcL$ as the strategy of player $i$. Following
standard game theoretic notation, all other players choose $L_{-i}$.
The expected total cost of player $i$ playing 
strategy $L_i$, with all other strategies fixed, is given by, 
$$
    \mathbb{E}\big(C(L_i,L_{-i})\big) = c\mathbb{E}\lbrack W(L_i,L_{-i})\rbrack + \hat c
    \mathbb{E}\big(\theta(L_i,L_{-i})\big),
$$
where $W(L_i,L_{-i})$ is the wait time, and
$c$ is the cost per unit waiting time, and $\hat c$ is the cost of sampling one queue. 
We define the fitness function, $\theta:\mcS \mapsto \mathbb{R}^n$, of player $i$ playing strategy $L_i$. 

Now, $\mcP(\mcL)$ is the family of sets containing all probability distributions over
all subsets of $\mcL$. The mixed
strategy $s_i(L_i) \in \mcS=\mcP(\mcL)$ is the probability that
eCR $i$ samples $L_i(\cdot)$ queues, 
the expected total cost for playing strategy $s_i$ is given by,
$$
    C(s_i, s_{-i}) = \displaystyle \sum_{L_i \in s_i} C(L_i,s_{-i})s(L_i).
$$
There exists a bijection from $\mcL$ to $\mcS=\mcP(\mcL)$. Let there be an
arrangement such that $s_i(L_i)$ be a non-decreasing step function in
$\mathbb{R}$, and normalizing factor $\pi \in [0,1]^n$ such that $\sum_{\mcL \in
s_i} \pi_i = 1$. In other words, given $m$ queues, 
define the indexed strategy set $\lbrace L_1 \le L_2 \le\cdots \le L_m\rbrace
\subset \mcL$ where the sum of the probabilities $\mathbb{P}(L_i)$ equals $1$.

We adopt the mean field theorem from Xu and Hajek (CITE).
Fix $s_i(\cdot) \in \mcS$, and suppose all other players use mixed strategy
$s_{-i}(\cdot)$. 
We introduce a transform $\gamma: \mcS \rightarrow \mcS$ such that,
$\gamma: s_i(\mcL) \rightarrow [0,1]^2$ such that,
$$
    \gamma(s_i \in \mcS) = \gamma\bigg(\displaystyle\sum_{s_1}^{s_i} \pi_{L_i} \bigg)-
    \gamma\bigg(\displaystyle\sum_{s_1}^{s_{i-1}}\pi_{L_i}\bigg) = 1.
$$
Let the strategy space $\mcS$ be the space of all cadlag ("continu a droite, limites a
gauches") functions; the space of right-continuous functions on $[0,1]$ with left
limits. Define a norm on $\mcS$, 
$$
    \vert\vert s_i \vert\vert = \sup_{L_i \in s_{\mcL_i}} \vert L_i(c)\vert.
$$
Define the variation process,
$$
    \langle s_i\rangle = \displaystyle\sum_{L_i \in s_{L_i}}
$$
Let $\Omega$ denote the class of strictly increasing continuous mappings of $[0,1]$ onto
itself. For $\pi \in \Omega$, $\pi \cdot \mcL \in [0,1]$, and we have a complete 
and separable metric space $\mcL, \mathbb{N}, \mcD(\mathbb{R},\mathbb{N})$.

In the mean field model, $\gamma$ defines cherry-picking and
time-constrained player types as separate distributions, and forms a partition
of the space where mixed strategy vector $[s]= [s_1, s_2, \cdots, s_n]$ is composed of
the strategies where $s_1 \le_{st} s_2 \implies \sum_{i = k, \cdots, n} s_1(L_i)
\le s_2(L_i) \ \forall \ k \in \mcL$. That is, $s_1$ is first order
stochastically dominated by $s_2$. We model the arrival rate of the eCRs $i$ as a 
Poisson Binomial distribution. Fix the mean $\mu$ of the arrival process so that
the variance of the distribution is bounded above by $\lambda = 1/\mu$.
The jump-diffusion process was constructed to have ergodic properties so that after initially flowing away from its initial condition it would generate samples from the posterior probability model.


Suppose $L(\cdot)\in \mcS$ is given, the solution to
$s_i(k) = \int_{c_i - k}^{c_i} \pi_i(c) \ dc$ gives a set of unique jumping points $0=c_0<
c_1< \cdots < c_{L_{MAX}} = c_{MAX}$. 
Any càdlàg finite variation process $X$ has quadratic variation equal to the sum
of the squares of the jumps of $X$.
Let $W_t: [0, +\infty) \times \Omega \rightarrow \mcS$ be a
one-dimensional geometric (exponential) Weiner process. We define right-continuity as 
a \emph{stopping time} $\tau:\mcS \rightarrow [0,+\infty]$, which occurs at a random jumping point. This is the first hitting time
for the region $\lbrace \hat c \in [0,\hat c_{MAX}] \vert c \ge c_max\rbrace$.
We have the following stochastic differential equation, with stopping time
$\tau(\omega)$,
$$
    dQ_t = \theta Q_t dt + \sigma Q_t dW_t,
$$
with Ito process,
$$
    \displaystyle\int_0^t \frac{dS_t}{S_t} = \mu t + \sigma W_t
    %S_t = S_0\exp\bigg(\big(\mu - \frac{\sigma^2}{2}\big) t + \sigma W_t\bigg).
$$
and $\mathbb{E}[S_t] = S_0e^{\mu t}$.
$$
    \gamma(s_i \in \mcS) = \gamma\bigg(\displaystyle\sum_{s_1}^{s_i} \pi_i \bigg)-
    \gamma\bigg(\displaystyle\sum_{s_1}^{s_{i-1}}\pi_i\bigg) = 1.
$$

A player's preference is thus modeled as cost coefficients assigned to the wait
time, $c$, and a cost $\hat c$, associated with the fitness of the queue.

\subsubsection{Deterministic Feedback}

We determine the fitness function as the action of a pair interaction between
players, following the replicator dynamics model for a haploid species. 
Thus, the fitness of a player is determined by the result of pair interactions,
and can be represented by the index $(i,j)^n$ in the
resulting $2^n$ matrix. Consider,
$$
    \theta(L_i,L_{-i}) = \displaystyle\sum_{j\in\L_{-i}}
    (1-L_i)(1-L_j)\theta)_{L_{-i}} + L_i L_j \theta_{i}.
$$
Let the private preferences of a player be defined by $\theta \in [0,1]^{L_i}$,
$$
    \theta(s_i, s_{-i}) = \displaystyle\sum_{L_i \in s_i} \theta_i(L_i, L_{-i})\theta_i.
$$
The deterministic feedback-based cost matrix is given as,
$$
    C(s_i, s_{-i}) = \displaystyle \sum_{L_i \in s_i} C(L_i,s_{-i})s(L_i).
$$
Let $s_{L_1} \le_{st} s_{L_2}$, indicate that $s_{L_1}$ is second-order stochastically
dominated by $s_{L_2}$, that is, 
$$
    \displaystyle\int_{-\infty}^s\vert F_{s_{L_1}}(t) - F_{s_{L_2}} \vert dt \ge 0.
$$
where $F_{s_\mcL}$ is the probability mass function of the
strategy space 
Denote this mapping by,
$$
    %GENERAL ENSEMBLE
$$
To illustrate,
consider a point $\zeta$ on the complex plane $\mathbb{C}$, and
let $N([s]$ be the set of neighbors; players who sample $L_i$ queues.
For a $2$-sphere 
of radius $r$, The mean field equation describes a separable and complete metric space
$(\mcS, \mathbb{R}^n, \mcD)$, where $\mcD$ is the space of right-continuous,
left-limited (cadlag) functions.



Let there be a partition function $Z:\mcS \times \mcS\rightarrow \mcS$, 
and fix the mixed strategy vector, $[s] = [s_1, \cdots, s_N] \in
\mcS\times\mcS$.
TODO: FIX THIS NONSENSE!\\
The probability that $[s]$ is the equilibrium state is,
$$
    \mathbb{P}([s]) = \displaystyle\sum_{s_i \in [s]} \frac{1}{Z} e^{-s
    (r_i^2)},
$$
with expected utility, $\mathbb{E}([s]) = \displaystyle\sum_{[s]} \pi_i
\mu(r_i)$.

Define a measure,
$$
    r_i = \min\lbrace \rho(s_i, s_j) : j\ne i\rbrace, \qquad (j = 1, \cdots N)
$$
where $r_i$ denotes the variation between $s_i$ and any other point in
$\mcS$. The open ball
$$
    S_i  = \lbrace s_i \in \mcS : \rho(s_i, s_j) < r_i, \ j\ne i \rbrace, \qquad (j = 1, \cdots N)
$$
is the sphere of influence $S_i \in \mcS\times\mcS$ over the strategy
space of player $i$, $s_i\circ\mcL = S_i = s_{\mcL} \in \mcS\times\mcS = \Pi_{L_i\in\mcL} s_{L_i}$. 


We model the eCR priority as the energy function, where the
time-constrained players are more energetic than the cherry-picking types, as
they have a higher incentive to sample more queues.
We have the $2$-sphere, TODO: collapse onto unit circle
$$
    \phi(S_i) = \delta\mu(S_i, S_{-i}) = 
$$
{\lemma{Convex Simplices}\\
We extend the arrangement $s_1<s_2<\cdots s_N$ to the trihedral plane. Fix
$s_{-i}\in \mcS$, and let the expected \emph{marginal value} of $L_i$ be,
$$
    V(L_i,s_{-i}) = \displaystyle\sum_{L_i\in s_{-i}\in \mcS}\mathbb{E}[C(L_i, L_{-i})
$$
Let $\mu$ be a step-function such that,
$$
    V(L_i,s_{-i}) \le \hat c/ c \le V(L_i-1, s_{-i}).
$$
Then, we claim $V(\cdot)$ forms a kernel of a convex space with a left limit.
The mapping exists if and only if there exists a strategy vector $[s]$, and a
function $\phi$ such that $\phi(s_i) = C((-\infty,s_i]), \ i = 1,2,\cdots, N$ and $\phi(s_1) < \phi(s_2) < \cdots < \phi(s_N)$, where phi is a skew-distribution, i.e. a
factorization of the skew-symmetric
matrix of the 4-vector, $[L_i]_{\times s_{-i}}$ that is also left-tailed.\\
}\\
\textbf{Proof:} 
Let $c_1 < c_2 \cdots < c_n$ be the costs associated with eCRs $\brace 1,\cdots,
n\rbrace$ in queue $k$. Fix time $t$ and let $\mu(\cdot)$ be defined as
$\max_{s_i\in Q}$ The inverse $\frac{1}{c_i} \approx 
\theta\mathbb{P(\mcS)}$. % UPPER BOUND
OUTLINE : (1) $c_k$ is the CDF of the $\mathbb{C}$ projection\\
(2) Per (CITE) we only need $\hat c$ to be cadlag, and both tails on the\\
$x$-axis (CAN WE CREATE THE SYMMETRY HERE?, NEED DYNAMICAL SYSTEM MODEL
($\phi$)\\
(3) Examine the convexity, concavity of the $\mcL$-forms (L-style junctions)!
Line can be obscured by overlapping objects, i.e. the overlapping radio range.
{\corollary{Skew-symmetry}
The map projection of $\mu_i\in \mcS$ onto the complex plane given by the Lorentz
translation,
$$ 
    \mu(s_i, s_{-i}) = \displaystyle\sum_{L_i \in [s_{-i}]_\times}
    \int_{c_{MIN}}^{c_{MAX}} V(L_i-1, s_{-i}),
$$
(NEED TO DEFINE STEP FUNCTION EXTENSION FOR VECTOR FIELD)
which we may interpret as the cumulative distribution function (CDF) across the
sample space $s_{\mcL} =\mcS\times \mcS$.

% TIKZ STACKED CUBES (figure)


We have the tangent line of the arrival process of eCRs sampling $L_i$ queues at price $c$
$$
    \lambda s_{-i}(L_i)(r_{L_i}(c-1) - r_{L_i}(c).
$$
Using the tangent lines, we begin a tesselation process on $SIG_i$, a collection of
mobius triangles which are oriented perpendicular to the direction of flux, and
define a map projection $\varphi(SIG(mathbb{B}(\cdot)))$ as a minimization of the Dirichlet energy, 
$$
	E[\mu]  = \frac{1}{2} \displaystyle\int_{\mcS}
    \vert\vert\triangledown\mu(s)\vert
    \vert^2 \ dV,
$$


We determine that the closed ball $\overline{B}_i$ belongs to the player for which 
$s_j<_{st} s_i \ \forall \ s_j \in [s]$.
Let $k$ be a queue containing $n$ eCRs and enveloping algebra $\gamma(S_i)$,
so that player $i$ holds the highest energy.
********** the flux conditions in three cases: (FINISH) is determined by the gradient
across the boundary of closed ball of $\overline{B}_i$.
Now define, for a player $i$, a function
$\alpha_t(\mcS) \longrightarrow S_i$ as as a mapping of queues in $\mcS$ to a
subset sampled by player $i$ at time $t$. 
Let $\mcM$ be a random measure on $(\mcS \times \mcS, \mcD)$, the $\sigma$-algebra of the
Poisson arrival process, and l

Define the set of best responses as a subset of $\mcS\times\mcS$ and a normalizing function $\mcG(S_i, S_{-i})$. 
We have 4 axioms; let balls $S_1, S_2$ be members of the stable state $\mcG$. 
As only pair interactions are used, we have transitivity, so for $S_1 < S_2 < S_3$, 
either $S_1 > S_2 \rightarrow S_1 > S_3$ or $S_2 > S_1 \rightarrow S_3 > S_1$, 
and we have completeness. Define the projection of the $n$-sphere onto the complex plane as a bijection $\mu: \mcL \rightarrow \mathbb{R}^n$ as $\mu(L_i, L_{-i}) = \theta$, where $\theta$ is a measure of the energy of the $n$-sphere an angle in $[0,2\pi]$ 

{\lemma{The Enveloping Algebra}\\
We extend the arrangement $s_1<s_2<\cdots s_N$ to the trihedral plane. Fix
$s_{-i}\in \mcS$, and let the \emph{marginal value} of $L_i$ be,
$$
    V(L_i,s_{-i}) = 
$$
if and only if there exists a strategy vector $[s]$, and a function $\phi$ such
that $\phi(s_i) =
C((-\infty,s_i]), \ i = 1,2,\cdots, N$ and $\phi(s_1) <
\phi(s_2) < \cdots < \phi(s_N)$. The union of these sets form an
algebra $[\alpha s] \subset \mcS\times\mcS$.
}\\
\textbf{Proof:} 
Let $c_1 < c_2 \cdots < c_n$ be the costs associated with eCRs $\brace 1,\cdots,
n\rbrace$ in queue $k$. Fix time $t$ and let $\phi(\cdot)$ be defined as
$\max_{s_i\in Q}$ The inverse $\frac{1}{c_i} \approx 
\theta\mathbb{P(\mcS)}$.
OUTLINE : (1) $\alpha_k$ is the CDF of 
We have the $2$-sphere, TODO: collapse onto unit circle

We take the cost function $C(\cdot)$ to be a nondegenerate
continuous probability density function
$$
    C(S_i,S_{-i}) = \int \frac{f(S)}{S - [s]_\times} dS
$$
where $f$ is any cadlag function.

Let $l$ be the tangent line to $B_i(\cdot)$,
$$
    l = \frac{(r+x)^2(r+y)^2}{r^2}.
$$

We may view the strategy space of eCR $i$ as a pair $(\mcL, \mathbb{P}(\mcS))$,
defining a distribution equipped with an orientation, as such we have a natural
case of the mapping function. 
We define a motivating operator $\phi$ to be a function of the probability
current, and a total variation distance defined by the player types,
$$
    \delta(s_i,s_{-i}) = \sup\limits_{\mcL \in s_{-i}} u(\mcL) - v(\mcL).
$$
$S_i$ may be represented by a statistical
ensemble (mixed type), with probability $p_i$ that the system is in a pure
state.
Define $\theta^*$ to be the complex
conjugate of the potential energy of the player in stasis, $\theta$.

{\lemma{The Hyperbolic Interior}\\ 
}
TODO: 
1. show gradient in each sphere from type distribution,
2. parameterize the hyperboloid
- use tangent space
3. complete the exterior, must be convex
4. define the incidents on the SIG


The stereographic coordinates of the upper sheet of a two-sheeted hyperboloid
in $\mathbb{R}^3$ are,
$$
    L = 
    \begin{cases}
    \phi = \frac{x}{1+z},\\
    \psi = \frac{y}{1+z}
    \end{cases}
$$
1. projection onto big sphere (exp)
2. limits being the map projection onto (sphere, hyperboloid)



Fix $s_i(\cdot) \in \mcL$, and suppose all other players use mixed strategy
$s_{-i}(\cdot)$, the expected fitness of queue $k$, 
$$
   C(s_i, s_{-i}) = c\displaystyle\sum_{[s]_i} G(L_i, s_{-i}) + \hat c L_i,
$$
where $G(L,s)\mapsto \mathbb{R}$ is a stereographic projection of the
exterior and interior of the manifold $\mcS(\mcL)$. 

We define the cost of an eCR joining a queue as a direct sum $G = g_s + g_\kappa$



\subsection{Enhanced Cognitive Structure}

The density matrix describing a mixed state is defined to be an operator of the form
$\rho =\sum _{s}p_{s}|\psi _{s}\rangle \langle \psi _{s}|$
where $p_{s}$ is the fraction of the ensemble in each pure
state $|\psi _{s}\rangle$ .
(this part needs the perceptron)
The ability of the CRN to reconfigure based on cognition of the environment
motivates us to construct our model as a study of momentum in a network between eCRs. 
We can think of this quantity as a measurement of flux passing through the
boundary of the ball $B_i$, although it is a positive quantity, the exterior of
the ball is a field with the same measure, and so although the field may have
velocity, there is no motivating factor as equilibrium is achieved at stasis.

We begin by examining the interactions between eCRs, the game begins without any
player holding any bias towards any other player, as no queues have been
sampled and no pair interactions have occured.
Thus we model the pair interaction, and define the the player's initial utility
function as a hyberbolic absolute risk aversion (HARA), and so must adhere to,
for utility $\mu$,
$$
    \displaystyle\frac{\mu''(C(L_i,L_{-i}))}{\mu'(C(L_i,L_{-i}))}.
$$
The initial utility for two eCRs is dependent on their type; cherry-pickers
(eCR$_{\omega}$), who
avoid the risk of oversampling, and time constrained eCRs, will try to
find the shortest queue, with higher costs (eCR$_{1-\omega}$).

We claim that there exists an additional, induced metric, on the
SIG. The the closed sphere of influence graph $SIG_i$ covers the intersections
of the closed balls $\mathbb{B}$.
$$
    \overline{B}_i= \lbrace X \in \mcX : \rho(X_i, X) \le r_i\rbrace.
$$
Consider the mapping $\mcS\mapsto \mcD$ that preserves the quadratic form $(t, r,
 \theta)\mapsto t^2-x^2-y^2-z^2$, and the general orthogonal group
 $O(1,3)$. As our group is a matrix \emph{Lie} group, and also a finite-dimensional
 smooth manifold. Let $\mcC^n$ be the extended complex plane, and so we have
 that $\mu(u,v) \mapsto u^{-1}v$ is a smooth mapping of the product manifold  $\mcS\times\mcS$ onto $\mcS$.


The quadratic form $\pm [c^2t^2 - x^2 -y^2 - z^2]$ can be used to define a bilinear form,
$$  
   u\cdot v = \pm [c^2t_1t_2 - r_1r_2],
$$
with orthonormal basis $-\eta(e_0,e_0) = \eta(e_1,e_1) = \eta(e_2,e_2) =
\eta(e_3,e_3) = 1$.
The function $\eta$ is the Minkowski $\eta(e_u, e_v) = \eta_{uv}$.

We define the closed graph $\overline{SIG_k}$ belonging to queue $k$.

\subsubsection{Sphere of Influence Graph (SIG)}
 We define the sphere of influence graph
(SIG) as a vertex set $\mcX$ and edge set 
$$
    \lbrace\lbrack s_i, s_j\rbrack : B_i \cap B_j \ne 0, i\ne j\rbrace.
$$
A SIG is \emph{simultaneously} a proximity graph and an intersection graph. 
We construct a map function and its closure in order to complete our
construction. The closed SIG graph forms a complete topology with respect to the
preference of its central eCR, 
within which we may make behavioral predictions. We define the SIG for an
eCR $i$, $SIG_i$ to be a set of vertices surrounding $i$, where another vertex
$j\in SIG_i$ probability of intersection is nonzero if and only if its core
$C_j$, is a subset of $SIG_i$ ((CHECK!). An intersection is defined to
be an interaction between the two eCRs (vertices) within a queue (CHECK!).
That is, the collection of balls who share an edge set at time $t$.
$$
    \mathbb{B} = \vert s_i \vert \le s_k \ \forall \ s_i \in S_i,
    s_k \in \mcS.
$$

Each ball represents a distribution of possible strategies, and as subset of the
measure space, we are able to compute its density function. 
Define the density operator $\rho$ on $\mcS\times \mcS$ as
$$
    \rho = \displaystyle\sum p_i\vert s_{L_i}\rangle\langle s_{L_i}\vert
$$
where $\vert s_i\rangle\langle s_i\vert$ is the outer product. The expectation
value of a state $[s]$ is given by $\langle [s] \rangle = \text{tr}{\rho [s]}$,


Consider a player $i$ with associated ball $B_i$ and graph $SIG_i$. Also consider that the ball $B_i$ has nonzero flux
at the boundery, and so $i$ is at the same time in an uncertain state. The
density matrix used here is defined to be the statistical state of a system in
quantum mechanics, and is particularly useful in dealing with mixed states.

We have an immersion in the surrounding
dynamical complex field $\mathbb{C}^2$. The complex field encases the distributions of the
player strategies; that is, the gradient of the distribution across the boundary
determines the orientation of exterior.
$$
    \theta
$$



TODO:
\begin{enumerate}
\item Create hyperbola in Minkowski space
\item Linearize node types using retraction (almost homogenous) using Lorentz
time-invariance 
\item Partition function to the product space
\item Start with identical particles (blank slate)
\item Define progression of belief (use the alleles?)
\item Mutate to generate attacks with fixed binomial dist (of attack type)
\end{enumerate}

\subsection{Evolved CRN (eCRN)}
CLAIM: We now show that the process cycle of the eCR adhers to the evolutionary
model. (evolved CRN)
\begin{itemize}
\item The manifold of the eCRs network is factorially bounded above by its
most exterior exponent, i.e. there exists a calibration on $\mcX$,
$$
\frac{\omega^k}{k!},
$$    
and therfore, the topology of the sublayer of the CRN is a simplectic
manifold $(\mcX,\omega)$. 
\item The eCRs are type invariant.
\item The exterior is continuous on the field.
\end{itemize}


The mobius triangles are subject to a piecewise skew transform, and
define a mapping function. We may assume that the the decision framework of
the eCRs is a skew function with respect to the flux conditions on the boundary
of
the surrounding field. 
{
\lemma{Existence and Contraction of a Kahler Manifold/ Mean Field Chaoticity}
}
\textbf{Proof:}
The fully connected and closed queueing system is fully
embedded in the frequency space. The emedding, along with the uncertainty
overlay allows us to formally represent the heterogenous eCRs. 
We will show that the decoupling of a distributed competitive game 
and the resulting niche market, a deformation of the space as the distribution
of density in the SIG manifold tends towards the central limit. Consider the
connected map generated on the skew plane, the flow of the phase space will
is diverted by the skew force, and may not lie on the outer hull, breaking
its convexity and introducing a nonlinear component to the smooth hull as
the tensor flow interacts with the surrounding field.
(FINISH)
\emph{Corollary:}
Every complex submanifold of a Kahler mainfold is volume minimizing in its
homology class.

TODO:
1. model the flux as the magnetic
component of the phase induced by the boundary potential (will be a plane)

The resulting topology is invariant under discrete possibility, and allows for
the weak convergence of our cadlag functor. Consider the infintesimal interval, 
$$
    dS^2 = dS_i^2 - dS_j^2,
$$
of strategies.

\section{Survivability}

In this section, we focus on vulnerabilites to denial-of-service (DoS) threats in the hostile network environment due to the uncertainties in the licensed user detection made during the spectrum sensing interval.
Our model assumes the existence of a hostile network environment, defined by the
presence our threat models, consisting of two types of attack: explicit or
implicit attacks. Explicit attacks include jamming, masking... Defense against
an explicit attack requires that the network detect an adversary hiding in the
PU spectrum. Implicit attacks are concerned with corruption of the decision
making process of the eCR. 
operating as an agent of the PU. As the adversary nodes are agents of the PU,
they are assumed to have knowledge of the characteristics of the PU's signal in
advance. The adverserial CPEs (ACPEs) therefore have an advantage as they do not
need to approximate the PU signal for detection. The adversary nodes are
classified according to their attacks and modeled as a Bayesian attack graph
(BAG). Describing the CRN similarly, we allocate defensive attributes to the eCR network graph, effectively forming an integrated defense system (IDS). We then form a Stackelberg game and derive bounds on the stability/ survivablility of the CRN.
W evaluate the performances of the IDS using the Positive Predictive Value (PPV), which is the probability that an alert raised by the IDS represents an actual intrusion. 
We begin with an introduction to the Stackelberg game from a topological
perspectiv game from a topological perspective, and show that a SOLUTION exists
and is a solution to the heirarchical game.

In particular, the overlapping areas of the individual CRN will have 
difficulty in enforcing security. 

\subsection{The Stackelberg Topology}

(EDIT)

\end{document}
