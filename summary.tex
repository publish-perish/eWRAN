\documentclass[10pt]{article}

\usepackage[text={6in,8.1in},centering]{geometry}
\usepackage{enumerate}
\usepackage{amsmath,amsthm,amssymb}
\usepackage{mathrsfs} % to use mathscr fonts
\usepackage{tikz}

\newcommand{\mcL}{\mathcal{L}}
\newcommand{\mcA}{\mathcal{A}}
\newcommand{\mcI}{\mathcal{I}}
\newcommand{\mcM}{\mathcal{M}}
\newcommand{\mcP}{\mathcal{P}}
\newcommand{\mcX}{\mathcal{X}}
\newcommand{\mcC}{\mathcal{C}}
\newcommand{\mcG}{\mathcal{G}}
\newcommand{\mcS}{\mathcal{S}}
\newcommand{\mcD}{\mathcal{D}}
\newcommand{\mcB}{\mathcal{B}}
\newcommand{\mcW}{\mathcal{W}}
\DeclareMathOperator*{\argmax}{arg\,max}
\DeclareMathOperator*{\argmin}{arg\,min}

\newenvironment{block}{\begin{adjustwidth}{1.5cm}{1.5cm}\noindent}{\end{adjustwidth}}

\newtheorem{proposition}{Proposition}[section]
\newtheorem{theorem}{Theorem}[section]
\newtheorem{lemma}{Lemma}[section]
\newtheorem{corollary}{Corollary}[section]
\theoremstyle{definition}
\newtheorem{definition}{Definition}[section]

\input pp.tex

\begin{document}

\section{Introduction}
In this work, we propose a mechanism to direct the distribution of contended
cells that will reveal the priority of a secondary cognitive radio user (CR).
Additionally, we model a learning process unique to the fluid and distributed
nature of the IEEE 802.22 network (WRAN). We model a cognitive radio network
(CRN) as a game among CRs, and maximize the utility of a set of overlapping base
stations (BS).
According to the 802.22 draft, the
self-coexistence quiet periods are used for the specific purpose to detect
overlapping WRANs, and are designed to support
dynamic resource sharing between the overlapping base stations (BSs), targeting
fair and efficient scheduling. We build our formulation on the specific case
where the BS cells must resort to adaptive on demand channel allocation. 
We examine a worst-case scenario, maximizing the
exploitation of uncertainties inherent to a spectrum sensing period.
Our focus is on the behavior of the cognitive radio during the
coexistence window, and their use of the coexistence beacon protocol. 
We provide an alternative
iteration of the process so that determines channels that can be acquired to satisfy the
QoS requirements of the given workloads. We determine that intra-channel sharing
is optimized by the cooperation, and the resulting inter-system demand
alliviates the channel contention processes with coexisting CRN. 
As network use in sparse areas becomes more widespread, it is likely that a central dispatcher will not be able
to provide the desired QoS across the distributed CRN. 

We consider the division of contended frames during the contention beacon
protocol. We propose a priority network policy, determined by a queuing network for frame
allocation. The CRs will choose the queue with the best priority match. 
We focus on three main issues in WRAN planning: (1) Spectrum allocation. A
network at equilibrium should maximize the throughput of the CRN through
intelligent use of base stations, (2) Quality of service (QoS). We consider QoS
to be a guarantee of a minimum rate of service while adhering to a priority
protocol. (3) Truthfulness in spectrum priority claims. 
We show that we are able to uphold
the desired property of self-coexistence, and model
a defense strategy as a Stackelberg game between the CRs and an adversary type
CR node with arrival rate and strategy as a directed mutations
of player type coefficients. Our goal is to maintain network coherence 
with the desired QoS converging to a dynamic, stable equilibrium. 

\section{Related Work}

A branch-and-bound (B/\&B) algorithm for IEEE 802.22-based LTE networks has been
developed in a queue-based control (QBC), where resource allocation is
controlled by the queue size of nodes following their packet arrival
probabilities. They achieve optimal power and resource block assignment for each
mobile user, trading execution time and end-to-end packet delay. (CITE) 

CR nodes are advancing technologically (Huawei), and will have the
(communications) resources to collaborate on the allocation of spectra in
addition to determining open channels, as in cooperative spectrum sensing
(CITE). 

Sakin and Razzaque solve the issue of self-coexistence as a large MIP problem,
which is known to be NP-hard, and several other approximations taking the
approach of nonlinear optimization (CITE).

With the 802.22 draft and the advent of orthogonal frequency modulation (OFM),
higher level algorithms began to form making use of the network capabilities.
Sengupta and Brahma (CITE) addressed the issues of efficiency and utilization
using a graph-theoretical approach. 

The mean field model was shown for general i.i.d random process, 
In [CITE] it was proven that there exists a Nash equilibrium in the
steady-state using Martingales.

\section{The Model}

\subsection{The Evolutionary Game}

In order to clarify our model, we use the CRN 
self-coexistence window header frame as an update window to
a type of central limit order book, which is owned and updated by the BS. 
The priority assignments from the eCRs waiting for contended frames 
from the BS are shared within the CRN via the eCRs within overlapping
areas. We form a cooperative game where the eCRs are responsible for forming queues in order
to optimize frame utilization of in the case where multiple BS must exist on the
same channel, defined in the 802.22 standard draft as on demand frame
contention. Our intention is to construct a queuing network at equilibrium, where the eCRs form
queues to wait for frames from a single, or multiple BS. 
We use a supermarket game, which is relevant in scenarios where (1) the players choose which
queue to join without directions from a central dispatcher; (2) global workload or
queue length information is not available and customers randomly choose a finite
number of queues based on knowledge from their nearest neighbors; (3) there is a cost
associated with finding and then waiting in queues.

We have a set of CRN, each containing a set of queues, one assigned to each BS. 
The queues formed at BSs that overlap cooperate intelligently. 
The 802.22 IEEE draft makes use of overlapping regions via the
contention beacon protocol. As in (CITE), we may assume that the BS randomly
selects a cluster head for each queue, who may aggragate or transmit data with
the BS and the other eCRs. Thus the importance of trust among queue members;
queues are natural targets for an adversary. 
The eCRs participate in a selection scheme that
encourages cooperation by a shared fitness function. The queues
are self-organizing and
self-enforcing; players may apply peer
pressure to abide by regulations by affecting queue fitness.

Following the lead of Lal
and Rao (1997), we define two types of eCR nodes: cherry pickers and
time-constrained. Cherry pickers are known to have a lower opportunity cost, and
prefer to maximize their utility in terms of sustainable bandwidth. 
Time-constrained players make up the majority, attributing a higher weight to
the cost of selecting a queue. 
In order to construct a viable model for our 
final goal, we must have (1) self-configuration, and (2) automatic neighbor relations. 
We address our CRs as a finite set of
actions, and formalize the game play as sphere-of-influence (SIG)(CITE) graph.
To begin, we propose a set of mixed
strategies defined by a probability distribution over
the finite set of feasible strategies, and define a 
supermarket game played by the eCRs.

\subsubsection{The Supermarket Game}

Consider a single CRN containing $M$ heterogenous FIFO queues with unit exponential service rate and global
Poisson arrival rate $\lambda$. A CR completing a transmission at queue $i$ will either 
move to some new queue $j$ with (fixed) probability $P_{ij}$ or leave the system 
with probability $\displaystyle 1-\sum _{j=1}^{m}P_{ij}$, which is non-zero for some subset of the queues. 
In order to arrive at an advanced decision model, we first address three main ways to describe the choice $L_i$, which serves to characterize the
supermarket game as a dynamic interaction of intelligent players: (1) Choice structure,
(2) Preference maximization, and (3) Utility maximization.
Player $i$'s choice of $L_i$ queues forms a preferred set, the interactions between the eCRs in forming their
preferred set of queues, and the interactions between eCRs belonging to a specific
queue form a decision profile. We define each queue of of eCRs as a coalition
within the CRN network including one or more BS, where the decision is based on
the expected probabilty distribution of the mixed strategies the eCRs.
We model the arrival rate of the eCRs $i$ as a 
Poisson binomial distribution, bounded above by $\lambda$.

The rules of the queue are as follows:
A queue cannot refuse a new member, however, we
consider that any queue member can submit a participation request to the BS.
Players joining the game choose a number of queues to be sampled uniformly at random and joins the best queue with respect
to the strategies of the other players. 

Denote a function $L_i(\cdot)$, from
$[0, c_{MAX}]$ to $\mcL$ as the strategy of player $i$. Following
standard game theoretic notation, all other players choose $L_{-i}$.
The expected total cost of player $i$ playing 
strategy $L_i$, with all other strategies fixed, is given by, 
$$
    \mathbb{E}\big(C(L_i,L_{-i})\big) = c\mathbb{E}\lbrack W(L_i,L_{-i})\rbrack + \hat c
    \mathbb{E}\big(\theta(L_i,L_{-i})\big),
$$
where $W(L_i,L_{-i})$ is the wait time, and
$c$ is the cost per unit waiting time, and $\hat c$ is the cost of sampling one queue. 
We define the fitness function, $\theta:\mcS \mapsto \mathbb{R}^n$, of player $i$ playing strategy $L_i$. 

Now, $\mcP(\mcL)$ is the family of sets containing all probability distributions over
all subsets of $\mcL$. The mixed
strategy $s_i(L_i) \in \mcS=\mcP(\mcL)$ is the probability that
eCR $i$ samples $L_i(\cdot)$ queues, 
the expected total cost for playing strategy $s_i$ is given by,
$$
    C(s_i, s_{-i}) = \displaystyle \sum_{L_i \in s_i} C(L_i,s_{-i})s(L_i).
$$
There exists a bijection from $\mcL$ to $\mcS=\mcP(\mcL)$. Let there be an
arrangement such that $s_i(L_i)$ be a non-decreasing step function in
$\mathbb{R}$, and normalizing factor $\pi \in [0,1]^n$ such that $\sum_{\mcL \in
s_i} \pi_i = 1$. In other words, given $m$ queues, 
define the indexed strategy set $\lbrace L_1 \le L_2 \le\cdots \le L_m\rbrace
\subset \mcL$ where the sum of the probabilities $\mathbb{P}(L_i)$ equals $1$.

We adopt the mean field theorem from Xu and Hajek (CITE).
Fix $s_i(\cdot) \in \mcS$, and suppose all other players use mixed strategy
$s_{-i}(\cdot)$. 
The cost matrix is given as,
$$
    C(s_i, s_{-i}) = \displaystyle \sum_{L_i \in s_i} C(L_i,s_{-i})s(L_i).
$$
Define $\kappa$ to be the number of queues of length at least $k$.
From (CITE), if $s_1,s_2\in\mcS$, with $s_1<_{st} s_2$, then
$\mathbb{E}[W(L,s_1)] > \mathbb{E}[W(L,s_2)]$.
The mean field equation describes a separable and complete metric space
$(\mcS, \mathbb{R}^n, \mcD)$, where $\mcD$ is the space of right-continuous,
left-limited (cadlag) functions.

Let $\Omega$ denote the class of strictly increasing continuous mappings of
$[0,1]$ onto itself. For $\pi \in \Omega$, $\pi \cdot \mcL \in [0,1]$, and we have a complete 
and separable metric space $\mcL, \mathbb{N}, \mcD(\mathbb{R},\mathbb{N})$.
In the mean field model, $[s]= [s_1, s_2, \cdots, s_n]$ is composed of
the strategies where $s_1 \le_{st} s_2 \implies \sum_{i = k, \cdots, n} s_1(L)
\le s_2(L) \ \forall \ \kappa \in \mcL$. That is, $s_1$ is first order
stochastically dominated by $s_2$. 
Fix $s_{-i}\in \mcS$, and let the expected \emph{marginal value} of $L_i$ be,
$$
    V(L_i,s_{-i}) = \displaystyle\sum_{L_i\in s_{-i}\in \mcS}\mathbb{E}[C(L_i,
    L_{-i})],
$$
where we define $\mathbb{E}[C(L_i,L_{-i})]$ 
as a stochastic process on a projective representation of the rotation group
$SO(3)$. The process $\kappa$ is determined by tensor flow.
Let $\kappa$ be a tensor, for cost $c, \hat c$, define,
$$
    \theta_i = \arctan\bigg(\frac{c}{\hat c}\bigg).
$$
The mapping exists if and only if there exists a strategy vector $[s]$, and a
function $\phi$ such that $\phi(s_i) = C((-\infty,s_i]), \ i = 1,2,\cdots, N$ and $\phi(s_1) < \phi(s_2) < \cdots < \phi(s_N)$, where phi is a skew-distribution, i.e. a
factorization of the skew-symmetric
matrix of the 4-vector, $[L_i]_{\times s_{-i}}$ that is also left-tailed.\\
\textbf{Proof:} TODO: %2
\textbf{Proof:} 
Let $\phi:\mcS\times\mcS \rightarrow \mcS$ be a function such that 
for cost $c$, 
$$
    \phi^{-1}(S_i) = \sup_{s_{L_i} \in S_i} [s]_\times = \begin{matrix} 0 & -t
    & \hat c \\ t & 0 & -c\\-\hat c & \omega & 0\end{matrix}
$$
To illustrate, let $n=2$ and consider $[s]= [s_1, s_2]$. Since $\pi \in [0,1]^n$
such that $\sum_{\mcL \in s_i} \pi_i = 1$ holds for all $s_i\in [s]$, and
$\mathbb{P}\big(\gamma([s])\big) = 1$, we have that $[s]_\times$ forms a basis for $S_i$. The function $\kappa_T$ determines the mean field equation
describing the number of queues of length at least $t$ with respect to the wait time, where, 
$$
    \displaystyle\int_{t=0}^\infty \kappa_{s_{-i}}^{L_i}(t) = \mathbb{E}[W(L_i, s_{-i})].
$$
% TIKZ STACKED CUBES (figure)
\begin{tikzpicture}
\planepartition{{5,3,2,2},{4,2,2,1},{2,1},{1}}
\end{tikzpicture}
We construct a bilinear form.
Let $c, \hat c \le 1$, and define a measure,
$$
    \mu_i = \min\lbrace \rho(s_i, s_j) : j\ne i\rbrace, \qquad (j = 1, \cdots
    n),
$$
and a labeling, $(1,2)\mapsto (c,\hat c)$, so that,
$$
    (\rho)^{-1}(S_i) = \sup_{s_{L_i} \in S_i} -\omega, \quad \begin{cases}
    (c,\hat c)
    = (1,2)\\ -1, \quad (c,\hat c) = (2,1) \\ 0 \quad c=\hat c\end{cases}.
$$
defines a mapping of a player's actions to a bifurcation, determined by the 
quadratic form $\pm [\omega t^2 - c^2 -\hat c^2 - \phi^2]$ 
surface $\rho(L_i) = c + \hat ci$, where $\phi(L_i) = r(t)e^{i\theta}$.
Suppose $\rho_i$ defines neighbors in $\mcS$, and uses a bilinear form,
$$  
   c\cdot \hat c = \pm [\omega t_1t_2 - r_1r_2],
$$
where $\omega = 1/\lambda$, and orthonormal basis $-\eta(e_0,e_0) = \eta(e_1,e_1) = \eta(e_2,e_2) =
\eta(e_3,e_3) = 1$, where the function $\eta$ is the skew-symmetric operator in
the Minkowski space.
We may extend the arrangement $s_1<s_2<\cdots s_N$ to the trihedral plane for
fixed time $\tau$. Although a mapping to the $n$-sphere is possible in theory,
we build from the geometric form, and extend the cross-product $[s_1]_\times
s_2$, which is characterized by the transition $L_i=k$ to $L_{i+1}=k+1$, forming
a tensor.
% TIKZ WEDGE STARS
Consider a mapping $\mcS\mapsto \mcD$ that preserves the quadratic form $(t, r,
\theta)\mapsto t^2-c^2-\hat c^2-\phi^2$, and the general orthogonal group $O(1,3)$.
The jump-diffusion process was constructed to have ergodic properties so that after initially flowing away from its initial condition it would generate samples from the posterior probability model.
We introduce a transform $\gamma: \mcS \rightarrow \mcS$ such that,
$\gamma: s_i(\mcL) \rightarrow [0,1]^2$ such that,
$$
    \gamma(s_i \in \mcS) = \gamma\bigg(\displaystyle\sum_{s_1}^{s_i} \pi_{L_i} \bigg)-
    \gamma\bigg(\displaystyle\sum_{s_1}^{s_{i-1}}\pi_{L_i}\bigg) = 1,
$$
Let $\mcD$ be the space of all cadlag ("continu a droite, limites a
gauches") functions; the space of right-continuous functions on $[0,1]$ with left
limits, and define a norm on $\mcS$, 
$$
    \vert\vert s_i \vert\vert = \sup_{c \in s_{\mcL_i}} \vert L_i(c)\vert.
$$
Define the variation process,
$$
    \langle s_i\rangle = \displaystyle\sum_{L_i \in s_{L_i}}
$$
TODO: %1
We proceed to determine the skew-distribution.
The map projection onto the complex plane is given by the Lorentz
translation,
$$
    f = -\epsilon \cdot s
$$
with derivative 
$$
    \frac{\partial f}{\partial s} = -\epsilon \cdot \frac{\partial s}{\partial
    s} = -\epsilon\cdot I = -\epsilon.
$$
We take the variation process of $S_i$ as, 
$$
    \langle S_i\rangle = \big\lbrace s \in S_{-i} \vert \sup_{s_{L_{-i}\times
    L_j}} <s_i, s_j> < \infty\big\rbrace.
$$
Now, let $\phi:\mcS\times \mathbb{N} \rightarrow S$ be a stochastic process
defining $dS_t$.
Suppose $L(\cdot)\in \mcS$ is given, the solution to
$s_i(k) = \int_{c_i - k}^{c_i} \pi_i(c) \ dc$ gives a set of unique jumping points $0=c_0<
c_1< \cdots < c_{L_{MAX}} = c_{MAX}$. 
Any cadlag finite variation process $S$ has quadratic variation equal to the sum
of the squares of the jumps $0=c_0<c_1</cdots$.
Let $W_t: [0, +\infty) \times \Omega \rightarrow \mcS$ be a
one-dimensional geometric (exponential) Weiner process. We define right-continuity as 
a \emph{stopping time} $\tau:\mcS \rightarrow [0,+\infty]$, which occurs at a random jumping point. This is the first hitting time
for the region $\lbrace \hat c \in [0,\hat c_{MAX}] \vert c \ge c_max\rbrace$.
We have the following stochastic differential equation, with stopping time
$\tau(\omega)$,
$$
    dS_t = \theta S_t dt + \phi S_t dW_t,
$$
with Ito drift-diffusion process,
$$
    \displaystyle\int_0^t \frac{dS_t}{S_t} = \theta t + \phi W_t
    %S_t = S_0\exp\bigg(\big(\mu - \frac{\sigma^2}{2}\big) t + \sigma W_t\bigg).
$$
and $\mathbb{E}[S_t] = S_0e^{\omega t}$.
Denote the mapping $(c, \hat c) \mapsto (r, t, \phi)$ by the open ball,
$$
    S = \lbrace s_i \in \mcS : \phi(s_i, s_j) < r_i, \ j\ne i \rbrace, \qquad (j
    = 1, \cdots N),
$$
where $\mathbb{E}[W(s_i, S_{-i}])] = r_i = \sqrt{c^2 + \hat c^2} \in [0,1]\times
[0,1]$, with transition given by the pairity of the arrival process,
$\text{sgn}(S_t) = (-1)^m$, so that $c_1<c_2$ implies $s_{L_1} >_{st} s_{L_2}$.
Now, $s_{L_1} \le_{st} s_{L_2}$, indicates that $s_{L_1}$ is second-order
stochastically dominated by $s_{L_2}$, that is, 
$$
    \displaystyle\int_{-\infty}^s\vert s_{L_1}(t) - s_{L_2} \vert dt \ge 0.
$$
is the sphere of influence $S_i \in \mcS\times\mcS$ over the strategy
space of player $i$, $s_i\circ\mcL = S_i \in \mcS\times\mcS$.
A player's preference is thus modeled as cost coefficients assigned to the wait
time, $c$, and a cost $\hat c$, associated with the fitness of the queue.

\subsection{Enhanced Cognitive Structure}

The ability of the CRN to reconfigure based on cognition of the environment
motivates us to construct our model as a study of momentum in a network between eCRs. 
We can think of this quantity as a measurement of flux passing through the
boundary of the ball $B_i$. Which we model as a time-stopped Weiner process.
TODO: FINISH, although it is a positive quantity, the exterior of
the ball is a field with the same measure, and so although the field may have
velocity, there is no motivating factor as equilibrium is achieved at stasis.

The stereographic coordinates of the upper sheet of a two-sheeted hyperboloid
in $\mathbb{R}^3$ are,
$$
    L = 
    \begin{cases}
    \phi = \frac{x}{1+z},\\
    \psi = \frac{y}{1+z}
    \end{cases}
$$
1. projection onto big sphere (exp)
2. limits being the map projection onto (sphere, hyperboloid)


{\lemma{The Hyperbolic Interior}\\ 
}
TODO: 
1. show gradient in each sphere from type distribution,
2. parameterize the hyperboloid
- use tangent space
3. complete the exterior, must be convex
4. define the incidents on the SIG




\end{document}
