\documentclass[10pt]{article}
\usepackage[text={6in,8.1in},centering]{geometry}
\usepackage{enumerate}
\usepackage{amsmath,amsthm,amssymb}
\usepackage{mathrsfs} % to use mathscr fonts
\newcommand{\mcL}{\mathcal{L}}
\newcommand{\mcI}{\mathcal{I}}
\newcommand{\mcM}{\mathcal{M}}
\newcommand{\mcP}{\mathcal{P}}
\newcommand{\mcX}{\mathcal{X}}
\newcommand{\mcC}{\mathcal{C}}
\newcommand{\mcS}{\mathcal{S}}
\newcommand{\mcD}{\mathcal{D}}
\DeclareMathOperator*{\argmax}{arg\,max}
\DeclareMathOperator*{\argmin}{arg\,min}
\newenvironment{block}{\begin{adjustwidth}{1.5cm}{1.5cm}\noindent}{\end{adjustwidth}}
\newtheorem{proposition}{Proposition}[section]
\newtheorem{theorem}{Theorem}[section]
\newtheorem{lemma}{Lemma}[section]
\newtheorem{corollary}{Corollary}[section]
\theoremstyle{definition}
\newtheorem{definition}{Definition}[section]

\begin{document}

\section{Introduction}

Game-theoretic ideas arise in many contexts. Often these settings are not called games, but they 
can be analyzed with the same tools. A decision-maker’s outcome depends on the decisions made by others. This 
introduces a strategic element that game theory is designed to analyze. However, game-theoretic 
ideas are also relevant to settings where no one is overtly making decisions. We
examine the behavior of agile and heterogenous nodes in a network environment, and model the
evolution of the player type under different network conditions and security
risks. 
Network security is a primary concern in open, 
dynamic and heterogeneous networks such as Internet and wireless mobile networks, 
mainly, research efforts are directed towards:
\begin{itemize}
    \item Characterizing the network model, incorporating constraints and
    underlying structure from well established quantative models.
	\item Characterizing malicious attacks incorporating the specific features of modern networks.
	\item Designing and refining (non)cooperative mechanisms.
	\item Designing and refining defense mechanisms.
\end{itemize}
Evolutionary biology provides an example. A basic principle is that mutations are more likely 
to succeed in a population when they improve the fitness of the organisms that carry the mutation. 
But often, this fitness cannot be assessed in isolation; rather, it depends on what all the other 
(nonmutant) organisms are doing and how the mutant’s behavior changes and
interacts within the network. In such situations, reasoning about the success or failure of the mutation involves 
game-theoretic definitions, and in fact very closely resembles the process of reasoning about 
decisions made by intelligent actors. 

The utilization of game theory to study the network security problems has attracted considerable 
research and has led to valuable insight on the attackers’ behaviour and the optimal strategy for 
the network defenders. A security game on a network 
is usually modeled by using a graph. We are motivated by modern network advances
to model network security problems with perspective from a game theoretic
analysis:
\begin{enumerate}
	\item Game theory is a powerful tool to model the 
interactions of decision makers with mutually conflicting / complimentary objectives.
 e.g., the interaction between the attackers and the network defenders /
 heterogenous utility functions and competing / contended nodes.
	\item Game theory (particularly non-cooperative game theory) can model the
    features or constraints of modern networks such as lack of coordination,
    network feedback and topology.
\end{enumerate}

Game theory can serve as a validation tool to evaluate the proposed solutions.
However, most of them are focused on the characterization of the Nash equilibrium (NE) of the formulated security game and the defenders’ strategy at the NE, few of them performs a systematic study on the
complexity (in terms of time and space) of how to solve the game and reach the
NE from a foundational game theory perspective. The proposed study aims at
filling this gap by establishing necessary theoretical foundations under a game algorithmic framework.



\section{Related work}

(Under construction...)

$[1]$ "Sphere of Influence Graphs in General Metric Spaces", T.S. Michael, T. Quint\\
$[2]$ "The Supermarket Game", Jiaming Xu, Bruce Hajek\\
$[3]$ "Modeling Population Dynamics in Changing Environments", Michelle Shen\\
$[4]$ "Self-coexistence among interference-aware IEEE 802.22 networks with enhanced air-interface", S. Sengupta, S. Brahma, M. Chatterjee, N. Sai Shankar\\

\section{Proposed Topic}


In order to construct a
viable model for our final goal, we must have (1) self-configuration, and (2) automatic neighbor
relations. We address our CRs as a finite set of actions, and formalize the game play as sphere-of-
influence (SIG) graph. To begin, we propose a set of mixed strategies defined by a probability
distribution over the finite set of feasible strategies, and define a basic game played by the
cognitive nodes.
Taking a queuing game, we have a basic set of strategies represented by a sampling of queues. This example is particularly useful in large, decentralized networks. The basic topology of the SIG graph under these rules allow for the projection of the decision process of the game into a higher-dimensional complex space. We analyze the interaction of the nodes as a topological graph, which we may then apply our statistical game theoretic approach. We do this by forming an arrangement. Given strategies $s_1, s_2$, and a function $\phi$, we examine the expanded strategy space where $s_1$ stochastic ally dominating $s_2$ implies that $E[\phi (\cdot, s_2 )] > E[\phi (\cdot, s_1 )]$. This space is well-publicized in marketing theory, where valuation functions are often given as right-continuous, left-limited functions. In decision theory, the von Neumann-Morgenstern utility theorem shows that, under certain axioms of rational behavior, a decision-maker faced with risky (probabilistic) outcomes of different choices will behave as if he or she is maximizing the expected value of some function defined over the potential outcomes at some specified point in the future. This function is known as the von Neumann-Morgenstern utility function. The theorem is the basis for expected utility theory.

As we expand the strategy space, new topologies emerge as a result of the SIG graph, allowing for the insertion of additional properties. We make use of mean field theorem, for which the study of arrangements is well-defined for random processes.

 to increase the complexity of the cognitive nodes' interaction space, and model it as a complex topology. 
The decision model in the extended strategy space is modeled as an Ito drift diffusion process, allowing us to make use of the Poisson process and binomial theorem. We conjecture that the choice of arrangement and random process to be in this space results in additional immersions of the SIG graph with nice properties.

Continuing, we model the additive noise to determine the rate of random mutation given in the evolutionary model. As we expand the strategy space
We conjecture that in this setting, we will be able to examine the interactions of the cognitive nodes, and model the evolution of their types. Given the goals of quality-of-service and robustness to adversary (mutant) nodes, we will be able to realistically determine the outcome of the game based on the intelligence of the nodes.

\section{Research Plan}


As an initial plan, the project will consist of the following main steps, which are also the main milestones of establishing the pertinent game algorithmic foundation on the network security problems.
\begin{itemize}
	\item Step 1: Study network security problems (literature study on the
    network security) and the relevant graph models. Formulate the network games using appropriate abstraction.
	\item Step 2: Establish the existence, and varying multiplicites, of the NE of the formulated security game. In case where the problem is NP complete, derive the relevant approximation or non-approximate results with distributed heuristic polynomial algorithms.
	\item Step 3: Study the programming of cognitive nodes, tensor flow.
	\item Step 4: Based on the analytical results, implement protocols and security solutions for various scenarios.
\end{itemize}

\section{Case Study I:}
In this work, we propose a mechanism to direct the distribution of contended cells that 
will reveal the priority of a secondary cognitive radio user (CR). Additionally, we model 
a learning process unique to the IEEE 802.22 network 
(WRAN). We model a cognitive radio network (CRN) as a game among CRs, and maximize the
 utility of a set of overlapping base stations (BS). According to the 802.22 draft, the 
self-coexistence quiet periods are used for the specific purpose to detect overlapping 
WRANs, and are designed to support dynamic resource sharing between the overlapping base
 stations (BSs), targeting fair and efficient scheduling. We build our formulation on 
the specific case where the BS cells must resort to adaptive on demand channel allocation.
 We examine a worst-case scenario, maximizing the exploitation of uncertainties inherent 
to a spectrum sensing period. Our focus is on the behavior of the cognitive radio 
during the coexistence window, and their use of the coexistence beacon protocol. 
We provide an alternative iteration of the process so that determines channels 
that can be acquired to satisfy the QoS requirements of the given workloads. We 
determine that intra-channel sharing is optimized by the cooperation, and the resulting 
inter-system demand alleviates the channel contention processes with coexisting CRN. 
As network use in sparse areas becomes more widespread, it is likely that a 
central dispatcher will not be able to provide the desired QoS across the distributed 
CRN.We consider the division of contended frames during the contention beacon protocol. 
We propose priority network policy, determined by a queuing network for frame 
allocation. The CRs will choose the queue with the best priority match. We focus on 
three main issues in WRAN planning: (1) Spectrum allocation. A network at equilibrium 
should maximize the throughput of the CRN through intelligent use of base stations, 
(2) Quality of service (QoS). We consider QoS to be a guarantee of minimum rate of 
service while adhering to a priority protocol. (3) Truthfulness in spectrum 
priority claims. We show that we are able to uphold the desired property of 
self-coexistence, and model defense strategy as a Stackelberg game between the CRs and 
an adversary type CR node with arrival rate and strategy as a directed mutations of 
player type coefficients. Our goal is to maintain network coherence with the desired 
QoS converging to a dynamic, stable equilibrium.

\subsection{Summary}
In order to arrive at an advanced decision model, we first
address three main ways to describe the choice $L_i$ , which serves to characterize the supermarket game
as a dynamic interaction of intelligent players: (1) Choice structure, (2) Preference maximization,
and (3) Utility maximization.

We define each queue of of eCRs as a coalition
within the CRN network including one or more BS, where the decision is based on the expected
probability distribution of the mixed strategies the eCRs. We model the arrival
rate of the eCRs $i$
as a Poisson binomial distribution, bounded above by $\lambda$.

\section{Plan for completion of research}

The criteria to evaluate the obtained work are:
\\ Theoretically, the success of the study is attested by the establishment of game algorithmic
foundation of the network security problems and the relevant algorithms developed to solve the
formulated security game and reach the NE of the game.
\\ Practically, the study is evaluated by the proposition of protocols and/or distributed defense
strategies based on the theoretic work. 

\section{The Extended Strategy Space}
\subsection{Case Study I:}
For strategy space $S$,
let $L_i$ be the number of subsets of $\mcS$ that $i$ chooses to sample.
Using standard game theory notation, for cost function $C$, we define $\mathbb{E}[C(L_i, L_{-i})]$ as a stochastic process on a projective representation of the rotation
group $SO(3)$. 
Define $\r(t)$ to be the number of queues of length at least $k$ at time $t$.
For $s_1, s_2 \in \mcS$, if $s_1,s_2\in\mcS$, with $s_1<_{st} s_2$, then
$\mathbb{E}[W(L,s_1)] > \mathbb{E}[W(L,s_2)]$.

We build from the geometric form, and
extend the cross-product $[s_1 ]_x s_2$ , which is characterized by the transition $L_i = k$ to $L_i+1 = k + 1$, forming a tensor field of similarity groups.
We examine the skew-Hermetian assignment, and the resulting tangent vector, or four-velocity. 
The four coordinate functions $\theta^a (\tau), a = 0, 1, 2, 3$ are real functions of a real variable $\tau$.
Consider a mapping $\mcS\mapsto \mcD$ that preserves the quadratic form $(t, r,
\theta)\mapsto (\omega^2 t^2-c^2-\hat c^2)$, and the general orthogonal group $O(1,3)$.
The jump-diffusion process was constructed to have ergodic properties so that
after flowing away from its initial condition it would generate
samples from the posterior probability model. We have that $\phi\cdot
\mcS\subset \mcS$ is the subset of skew-Hermetian matrices known as signed
permutation matrices. We extend our mean-field model to include this
representation by setting $\begin{bmatrix}0 // k\end{bmatrix} = \begin{bmatrix}1 //
k\end{bmatrix}$ at the stop time $\tau$.
We proceed to determine the skew-distribution. The map projection onto the complex plane is given by the Lorentz translation, which gives the displacement $L_i = k$ to $L_i+1 = k +1$ as a function of time.
Let $\mcD$ be the space of all cadlag (”continu a droite, limites a gauches”) functions; the space of
right-continuous functions on $[0, 1]$ with left limits.
Now, let $\phi:\mcS\times \mathbb{N} \rightarrow S$ be a stochastic process
defining(?) $dS_t$.
Suppose $L(\cdot)\in \mcS$ is given, the solution to
$s_i(k) = \int_{c_i - k}^{c_i} \pi_i(c) \ dc$ gives a set of unique jumping points $0=c_0<
c_1< \cdots < c_{L_{MAX}} = c_{MAX}$. 
Any cadlag finite variation process $S$ has quadratic variation equal to the sum
of the squares of the jumps $0=c_0<c_1<\cdots c_n$.
Let $W_t: [0, +\infty) \times \Omega \rightarrow \mcS$ be a
one-dimensional geometric (exponential) Weiner process. We define right-continuity as 
a \emph{stopping time} $\tau:\mcS \rightarrow [0,+\infty]$, which occurs at a random jumping point. This is the first hitting time
for the region $\lbrace \hat c \in [0,\hat c_{MAX}] \vert c \ge c_{max}\rbrace$.
We have the following stochastic differential equation, with stopping time
$\tau(\cdot)$,
$$
    dS_t = \theta S_t dt + \phi S_t dW_t,
$$
with Ito drift-diffusion process,
$$
    \displaystyle\int_0^t \frac{dS_t}{S_t} = \theta t + \phi W_t
$$
A player’s preference is modeled as cost coefficients assigned to time, $c$, and a cost $\hat c$, associated with the fitness of the coalition.
Denote the mapping $(c, \hat c) \mapsto (r, t, \phi)$ by the open ball,
$$
    S = \lbrace s_i \in \mcS : \phi(s_i, s_j) < r_i, \ j\ne i \rbrace, \qquad (j
    = 1, \cdots N),
$$
where $\mathbb{E}[C(s_i, S_{-i}])] = r_i = \sqrt{c^2 + \hat c^2} \in [0,1]\times
[0,1]$.
We intend to show that $s_{L_1} \le_{st} s_{L_2}$ indicates that $s_{L_1}$ is second-order stochastically dominated by $s_{L_2}$, that is, 
$$
    \displaystyle\int_{-\infty}^s\vert s_{L_1}(t) - s_{L_2} \vert \ dt \ge 0.
$$
is the sphere of influence $S_i \in \mcS\times\mcS$ over the strategy
space of player $i$, $s_i\circ\mcL = S_i \in \mcS\times\mcS$.

Now, the sphere of influence (SIG) is defined to be a vertex set with an edge joining
a pair of distinct vertices provided the corresponding spheres of influence intersect. This graph was introduced by Toussaint to model computer vision and pattern recognition problems in the
Euclidean plane, and is simultaneously a proximity graph and an intersection graph.

Define the the player's initial utility
function as a hyperbolic absolute risk aversion (HARA), and so must adhere to,
for utility $\mu$,
$$
    \displaystyle\frac{\mu''(C(L_i,L_{-i}))}{\mu'(C(L_i,L_{-i}))}.
$$
For a random measure on $(\mcS \times \mcS, \mcD)$, the $\sigma$-algebra of the
Poisson arrival process, we have that a NE exists in the queuing network, where we define the SIG as the set subsets of $\mcS\times\mcS$ of fully connected nodes.
Each ball $\mcS_i$ represents a distribution of possible strategies, and as subset of the
measure space we are able to compute its density function. 
Define the density operator $\rho$ on $\mcS\times \mcS$ as
$$
    \rho = \displaystyle\sum p_i\vert s_{L_i}\rangle\langle s_{L_i}\vert
$$
where $\vert s_i\rangle\langle s_i\vert$ is the outer product. The expectation
value of a state $[s]$ is given by $\langle [s] \rangle = \text{tr}{\rho [s]}$,

Consider a player $i$ with associated ball $B_i$ and graph $SIG_i$. Also consider that the ball $S_i$ has nonzero flux
at the boundary, and so $i$ is at the same time in an uncertain state. The
density matrix used here is defined to be the statistical state of a system in
quantum mechanics, and is particularly useful in dealing with mixed states.

We have an immersion in the surrounding
dynamical complex field $\mathbb{C}^2$. The complex field encases the distributions of the
player strategies; that is, the gradient of the distribution across the boundary
determines the orientation of exterior.
We claim that there exists an additional, induced metric, on the
SIG. The the closed sphere of influence graph (CSIG) covers the intersections
of the closed balls $\lbrace\overline{\mcS}\rbrace \subset \mcS\times \mcS$, where
$$
    \overline{S}_i= \lbrace S \in \mcS : \min\rho(S_i, S_{-i}) \le r_i\rbrace.
$$
Forming the CSIG,
we endow the resulting strategy space with the Minkowski metric $\eta$.
Minkowski space $\mcM$ is not endowed with a Euclidean geometry, and not with any of the generalized Riemannian geometries with intrinsic curvature. The reason is the indefiniteness of the Minkowski metric. Minkowski space is not a metric space and not a Riemannian manifold with a Riemannian metric, Minkowski space contains submanifolds endowed with a Riemannian metric yielding hyperbolic geometry. The Minkowski metric, also called the Minkowski tensor or pseudo-Riemannian metric, is a tensor $\eta_{\alpha\beta}$ whose elements are defined by the matrix
$$
	(\eta)_\alpha\beta = \begin{bmatrix}
		-1 & 0 & 0 & 0 \\
		0 & 1 & 0 & 0 \\
		0 & 0 & 1 & 0 \\
		0 & 0 & 0 & 1 \\
	\end{bmatrix},
$$
where the indices $\alpha,\beta$ run over $0, 1, 2,$ and $3,$ with $x^0=t$ the time coordinate and $(x^1,x^2,x^3)$ the space coordinates. This is simply the SIG where triangles are now hyperbolic triangles. 
Each ball represents a distribution of possible strategies, and as subset of the
measure space, we are able to compute its density function. 
Now, CSIGs always have a clique factor, that is CSIG $G = \lbrace G_1,\cdots G_n\rbrace$ has a spanning subgraph whose connected components are always isomorphic to graphs in the set $\lbrace G_1,\cdots G_n\rbrace$. These connected components define the incident matrices of our dynamical network system. For example, suppose the node set $\mcX$ realizes $G$ in $\mcM$. Select a proper spanning forest $F$
of the $\mcM$-CSIG such that the sum of the lengths of the edges of $F$ is minimum, then, each connected component of $F$ must be a star.

We will define the priority of each member of the subset $\lbrace\overline\mcS\rbrace$ based on this clique topology, so that each subset of strategies represents an $n$-dimensional simplex. The fitness of each strategy is associated with the type of node and its relative position in the clique. The clique members will replicate randomly according to the Ito process, with mutations occurring at a rate given by the noise process.
We define a diffusion process using the replicator equation, and determine the
fitness function as the pair interaction between players, following the replicator dynamics model for a haploid species. 
Thus, the fitness of a player is determined by the result of pair interactions,
and can be represented by the index $(i,j)^n$ in the
resulting $2^n$ matrix. Consider,
$$
    \theta(L_i,L_{-i}) = \displaystyle\sum_{j\in\L_{-i}}
    (1-L_i)(1-L_j)\theta)_{L_{-i}} + L_i L_j \theta_{i}.
$$
Let the private preferences of a player be defined by $\theta \in [0,1]^{L_i}$,
$$
    \theta(s_i, s_{-i}) = \displaystyle\sum_{L_i \in s_i} \theta_i(L_i, L_{-i})\theta_i.
$$
As our group is a matrix \emph{Lie} group, and also a finite-dimensional
smooth manifold, and so we have
that $\mu(S_i, S_{-i}) \mapsto s_i^{-1}s_{-i}$ is a smooth mapping of the
product manifold  $\mcS\times\mcS$ onto $\mcS$.

The utility function is described by the geodetic lines of the dominant strategy $S_i$.
{\lemma{The Enveloping Algebra}\\
We extend the arrangement $s_1<s_2<\cdots s_N$ to the trihedral plane. Fix
$s_{-i}\in \mcS$, and let the \emph{marginal value} of $L_i$ be,
$$
    V(L_i,s_{-i}) = \mathbb{E}[\phi(L_i, s_{-i})] - \mathbb{E}[\phi(L_i+1,
    s_{-i})].
$$
We suppose(?) there exists a strategy vector $[s]$, and a function $\phi$ such
that $\phi(s_i) =
C((-\infty,s_i]), \ i = 1,2,\cdots, N$ and $\phi(s_1) <
\phi(s_2) < \cdots < \phi(s_N)$. The union of these sets form an
algebra $[\alpha s] \subset \mcS\times\mcS$.
}\\
\textbf{Proof:} 
Let $c_1 < c_2 \cdots < c_n$ be the costs associated with eCRs $\brace 1,\cdots,
n\rbrace$ in queue $k$. Fix time $t$ and let $\phi(\cdot)$ be defined as
$\max_{s_i\in Q}$ The inverse $\frac{1}{c_i} \approx 
\theta\mathbb{P(\mcS)}$.
OUTLINE : (1) Define, 
$$ 
    (s_i, s_{-i}) = \displaystyle\sum_{L_i \in [s_{-i}]_\times}
    \int_{c_{MIN}}^{c_{MAX}} V(L_i-1, s_{-i}),
$$
which we interpret as a sample of the cumulative distribution function (CDF)
across the sample space $s_{\mcL} =\mcS\times \mcS$.\\

TODO (define model): \\
1. form the mobius negative binomial stitching\\
2. attach to geodesics (define geodesics, looking for isometry, not
diffeomorphism)\\
3. axis-align to random process with cone to infty\\

TODO (define priority/utility constraints): 
1. show gradient in each sphere from type distribution\\
2. parameterize the hyperboloid to define tangent space\\
3. complete the exterior, must be convex\\
4. define the incidents on the SIG\\

TODO (define game):\\
1. Determine subset of best responses\\

\section{Research Plan}


As an initial plan, the project will consist of the following main steps, which are also the main milestones of establishing the pertinent game algorithmic foundation on the network security problems.
\begin{itemize}
	\item Step 1: Study network security problems (literature study on the
    network security) and the relevant graph models. Formulate the network games using appropriate abstraction.
	\item Step 2: Establish the existence, uniqueness if the case, of the NE of the formulated security game. In case where the problem is NP complete, derive the relevant approximation or non-approximability results with distributed heuristic polynomial algorithms.
	\item Step 3: Study the programming of cognitive nodes, tensorflow.
	\item Step 4: Based on the analytical results, implement protocols and security solutions for various scenarios.
\end{itemize}


\section{Plan for completion of research}

The criteria to evaluate the obtained work are:
\\ Theoretically, the success of the study is attested by the establishment of game algorithmic
foundation of the network security problems and the relevant algorithms developed to solve the
formulated security game and reach the NE of the game.
\\ Practically, the study is evaluated by the proposition of protocols and/or distributed defense
strategies based on the theoretic work.

\section{Literature}

In addition to related work, the sourced literature must include historical papers to give perspective and formalization, and determine the foundation of the proposed theory.
$[1]$ M. E. Bratman. Intention, Plans, and Practical Reason. CSLI Publications, Stanford University,
1987.\\
$[2]$ I. Gilboa and E. Zemel. Nash and correlated equilibria: Some complexity considerations. Games
and Economic Behavior, 1:80-93, 1989.\\
$[3]$ E. Kalai. Games, computers, and O.R. In ACM/SIAM Symposium on Discrete Algorithms, 1995.\\
$[4]$ Noam Nisan and Amir Ronen. Algorithmic mechanism design. In Proc. 31st ACM Symp. on
Theory of Computing, pages 129-140, 1999.\\
$[5]$ C. H. Papadimitriou. On the complexity of the parity argument and other inefficient proofs of
existence. Journal of Computer and System Sciences, 48(3):498-532, 1994.\\
$[6]$ J. von Neumann and O. Morgenstern. Theory of Games and Economic Behavior, Second Edition.
Princeton University Press, second edition, 1947.\\

\end{document}

